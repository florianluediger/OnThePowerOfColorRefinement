%------------------------------------------------
% Latex-Grundgerüst für Seminarausarbeitungen
%
% zu Erzeugen mit
%   pdflatex ausarbeitung.tex
%
% von Carsten Gutwenger, Nils Kriege
% 
%------------------------------------------------

\documentclass[a4paper,12pt,twoside]{article}
\usepackage[a4paper,left=3.5cm,right=2.5cm,bottom=3.5cm,top=3cm]{geometry}

% Encoding der .tex-Datei
% muss je nach Editor-Einstellungen eventuell angepasst werden!
%   latin1    Unix/Windows
%   ansinew   Alternative für Windows
%   applemac  Apple
%   utf8      Alternative für Unix/Windows/Apple
\usepackage[utf8]{inputenc}

\usepackage[ngerman]{babel}                                 % deutsche Sprache
\usepackage[T1]{fontenc}                                    % Unterstützung für Umlaute mit Fonts
\usepackage[pdftex]{graphicx,color}                         % Einbetten von Grafiken, Farbe
\usepackage[format=plain,small,bf,indention=7.5mm]{caption} % verbesserte Beschriftung von Abbildungen
\usepackage{subfig}                                         % subfigures
\usepackage{url}                                            % \url{}-Kommando
\usepackage{fancyhdr,float}                                 % Kopf-/Fußzeilen
\usepackage[pdftex,pdfpagelabels]{hyperref}                 % Hyperlinks im PDF
\usepackage[round]{natbib}                                  % Literaturverzeichnis
\usepackage{doi}                                            % DOIs
\usepackage[section,boxed]{algorithm}                       % Floating-Umgebung für Algorithmen
\usepackage{algpseudocode}                                  % Pseudocode für Algorithmen
\usepackage[standard]{ntheorem}                             % Theorem-Umgebung
\usepackage{mathtools}										% Tools zum Anzeigen von Formeln
\usepackage[nogroupskip,style=altlist]{glossaries}			% Verwaltung des Glossars
\usepackage{enumitem}										% Erweiterte Funktionen der enumerate Umgebung

\captionsetup{farskip=10pt,topadjust=0pt,captionskip=10pt,nearskip=0pt,margin=10pt}

\pagestyle{fancy}
\fancyhead[LE,RO]{\thepage}
\fancyhead[RE]{\nouppercase{\slshape \leftmark}}
\fancyhead[LO]{\nouppercase{\slshape \rightmark}}
\fancyfoot[C]{}

% Erstellen des Glossars
\setacronymstyle{long-short-desc}
\renewcommand{\glstextformat}[1]{\textit{#1}}
\makenoidxglossaries

% Definieren von mathematischen Symbolen
\newcommand*{\ldblbrace}{\{\mskip-5mu\{}	% {{
\newcommand*{\rdblbrace}{\}\mskip-5mu\}}	% }}

%\newglossaryentry{}{name={},description={}}
%\newacronym[description={}]{label}{kurz}{lang}
\newglossaryentry{regulaerer_graph}{name={regulärer Graph},description={Ein Graph ist regulär wenn alle seine Knoten den gleichen Grad besitzen}}
\newglossaryentry{unigraph}{name={Unigraph},description={Die Isomorphieeigenschaften eines Unigraphen sind durch die Sequenz der Knotengrade genau definiert. Dies bedeutet, dass allein anhand der Knotengrade zweier Unigraphen bestimmt werden kann, ob diese isomorph sind}}
\newglossaryentry{nachbarschaft}{name={Nachbarschaft},description={Die Nachbarschaft $N(u)$ bildet die Menge der Knoten, die adjazent zu $u\in V(G)$ sind},plural={Nachbarschaften}}
\newglossaryentry{subgraph}{name={Subgraph},description={Der Subgraph $G[X]$ ist ein Teilgraph von $G$, der durch die Knotenmenge $X\subseteq V(G)$ und deren inzidenten Kanten gebildet wird}}
\newglossaryentry{bipartiter_graph}{name={bipartiter Graph},description={Ein Graph heißt bipartit, wenn sich seine Knoten in zwei Teilmengen aufteilen lassen, sodass Kanten nur zwischen den beiden Mengen aber nicht innerhalb existieren. $G[X,Y]$ ist der bipartite Graph, welcher durch die beiden disjunkten Teilmengen $X,Y\subseteq V(G)$ und allen Kanten, die Knoten aus $X$ und $Y$ verbinden, gebildet wird}}
\newglossaryentry{disjunkte_vereinigung}{name={disjunkte Vereinigung},description={Die knotendisjunkte Vereinigung von $G$ und $H$ wird $G+H$ genannt. Die disjunkte Vereinigung von $m$ Kopien des Graphen $G$ wird als $mG$ geschrieben}}
\newglossaryentry{bipartites_komplement}{name={bipartites Komplement},description={Das bipartite Komplement eines Graphen $G$ mit Knotenklassen $X$ und $Y$ stellt der bipartite Graph $G'$ dar, welcher die selben Knotenklassen enthält, allerdings das Komplement der Kanten zwischen den beiden Knotenklassen}}
\newglossaryentry{biregulaerer_graph}{name={biregulärer Graph},description={Ein bipartiter Graph $G$ mit Knotenklassen $X$ und $Y$ ist biregulär, wenn alle Knoten in $X$ und $Y$ den gleichen Grad besitzen}}
\newglossaryentry{vollstaendiger_graph}{name={vollständiger Graph},description={In einem vollständigen Graphen $K_n$ mit $n$ Knoten, ist jeder Knoten mit jedem anderen Knoten verbunden und besitzt somit den Grad $n-1$}}
\newglossaryentry{vollstaendiger_bipartiter_graph}{name={vollständiger bipartiter Graph},description={In einem vollständigen bipartiten Graphen mit den Knotenmengen $X$ und $Y$ sind alle Knoten aus $X$ mit allen Knoten aus $Y$ verbunden. Somit haben alle Knoten aus $X$ den Grad $|Y|$ und alle Knoten aus $Y$ den Grad $|X|$}}
\newglossaryentry{zyklus}{name={Zyklus},description={Ein geschlossener Pfad eines Graphen über $n$ Knoten wird Zyklus $C_n$ genannt}}
\newglossaryentry{multimenge}{name={Multimenge},description={Eine Multimenge unterscheidet eine Menge dadurch, dass Elemente mehrfach vorkommen können. Multimengen werden hier durch doppelte geschweifte Klammern ${{}}$ dargestellt},plural={Multimengen}}
\newglossaryentry{matching_graph}{name={matching Graph},description={Ein matching Graph ist ein Graph, bei dem kein Knoten mehr als eine inzidente Kante besitzt. Somit gibt es nur Zusammenhangskomponenten mit maximal einer Kante und zwei Knoten}}
\newglossaryentry{hypergraph}{name={Hypergraph},description={Ein Hypergraph ist ein Graph, in dem eine Kante, auch Hyperkante genannt, mehr als zwei Knoten verbinden kann}}
\newglossaryentry{stern}{name={Stern},description={In einem Sterngraphen $K_{1,t}$ gibt es einen zentralen Knoten, welcher mit allen anderen Knoten des Graphen, $t$ an der Zahl, durch eine Kante verbunden ist. Die anderen Knoten sind untereinander nicht verbunden}}
\newglossaryentry{disjunkte_vereinigung_von_sternen}{name={disjunkte Vereinigung von Sternen},description={Bei einer \glslink{disjunkte_vereinigung}{disjunkten Vereinigung} von \glslink{stern}{Sternen} $sK_{1,t}$ bezeichnet $s$ die Anzahl der Sterne und $t$ die Anzahl der äußeren Knoten jedes Sterns}}

\begin{document}

%------------------------------------------------
% Titelseite
%------------------------------------------------

\begin{titlepage}
\vspace*{-2cm}
\newlength{\links}
\setlength{\links}{-1.5cm} \sf \LARGE

\hspace*{\links}
\begin{minipage}{12.5cm}
\includegraphics[width=8cm]{tud_logo_rgb}
\end{minipage}

\vspace*{4cm}

\large
\begin{center}
{\Large Seminarausarbeitungskonzept} \\[1ex]
{\LARGE\textbf{On the Power of Color Refinement}}\\[3ex]
Florian Lüdiger\\[1ex]
\today\\[7ex]
im Rahmen des Seminars\\[1ex]
{\Large\textbf{Algorithm Engineering}}\\[1ex]
von Prof.~Dr.~Petra Mutzel\\[1ex]
Wintersemester 2017/18
\end{center}

\vspace*{4cm}
\hspace*{\links}
\begin{minipage}[b]{15cm}
\normalsize \raggedright
\textbf{Betreuer:} \\
Christopher Morris \\[2ex]

\textbf{Basierend auf:}\\
V. Arvind, Johannes Köbler, Gaurav Rattan und Oleg Verbitsky\\
On the Power of Color Refinement\\
\url{https://link.springer.com/chapter/10.1007/978-3-319-22177-9_26}
\end{minipage}

\definecolor{TUGreen}{rgb}{0.517,0.721,0.094}
\vfill
\hspace*{\links}
\begin{minipage}[b]{8cm}
\normalsize \raggedright
Fakultät für Informatik\\
Algorithm Engineering (Ls11)\\
Technische Universität Dortmund
\end{minipage}

\end{titlepage}

%------------------------------------------------
% Inhaltsverzeichnis
%------------------------------------------------

\tableofcontents
\clearpage

%------------------------------------------------
% Glossar
%------------------------------------------------
\glsaddall
\printnoidxglossaries

%------------------------------------------------
% Kapitel
%------------------------------------------------

\section{Einführung}
\label{sec/einfuehrung}

\begin{itemize}
	\item Allgemeine Informationen zum Paper
	\item Was ist das Ziel der Forschung und inwiefern wurde dieses erreicht?
	\item Überblick über die Kapitelstruktur
\end{itemize}


\section{Graph-Isomorphie und Color Refinement}


\section{Lokale Struktur von CR-Graphen}
\label{sec/struktur_lokal}

%\begin{itemize}
%	\item Erklärung der lokalen Struktur zugänglicher Graphen (A,B)
%	\item jeweils gegebenenfalls mit ausführlicher Erklärung der Bedeutung und Beispielen
%	\item Fokussierung auf die Beweisidee für die jeweiligen Lemmata
%	\item Ausführlichere und leichter verständliche Aufbereitung einiger oder aller Beweise aus dem Paper
%\end{itemize}

Um schlussendlich die Frage beantworten zu können von welcher Beschaffenheit CR-Graphen sein müssen, damit sie die in Definition \ref{def:cr-graph2} genannte Eigenschaft erfüllen, werden hier zunächst notwendige, lokale Eigenschaften solcher Graphen vorgestellt.
Die Basis dieses Kapitels bildet das folgende Lemma, welches für beliebige Zellen $X$ und $Y$ der stabilen Partition $\mathcal{P}_G$ eines CR-Graphen $G$ einige Merkmale definiert.

\begin{Lemma}
	Die Zellen der stabilen Partition $\mathcal{P}_G$ eines CR-Graphen erfüllen folgende Eigenschaften:
	
	\begin{enumerate}[label=(\Alph*)]
		\item Für beliebige Zellen $X\in \mathcal{P}_G$ ist $G[X]$ ein leerer Graph, \gls{vollstaendiger_graph}, \gls{matching_graph} $mK_2$, das Komplement eines matching Graphen oder der 5er \gls{zyklus}
		\item Für beliebige Zellen $X,Y\in \mathcal{P}_G$ ist $G[X,Y]$ ein leerer Graph, \gls{vollstaendiger_bipartiter_graph}, eine \gls{disjunkte_vereinigung_von_sternen} $sK_{1,t}$, bei der $X$ die Menge der $s$ inneren Knoten und $Y$ die Menge der $st$ Blätter ist, oder das \glslink{bipartites_komplement}{bipartite Komplement} des zuletzt genannten Graphen
	\end{enumerate}
	\label{lemma:lokal}
\end{Lemma}

Ein leerer Graph beschreibt in diesem Falle einen Graphen, welcher zwar Knoten allerdings keine Kanten enthält.
Wichtig ist an dieser Stelle außerdem, dass die in \emph{B} angesprochene Vereinigung von Sternen ausschließlich Sterne enthält, die die gleiche Anzahl, nämlich $t$, von Blättern besitzen.
Um das Lemma beweisen zu können, werden zunächst folgende Hilfsaussagen benötigt.

\begin{Lemma}
	Ein regulärer Graph mit Grad $d$ und $n$ Knoten ist ein \gls{unigraph}, genau dann wenn $d\in \{0,1,n-2,n-1\}$ oder $d=2$ und $n=5$.
	\label{lemma:lokal_regulaer}
\end{Lemma}

Der Beweis zu dieser Aussage findet sich in \cite{johnson1975simple}.
Bei genauerer Betrachtung fällt auf, dass die in \emph{A} genannten Graphentypen sich in diesem Lemma wieder finden. Tabelle \ref{tab:mapping_regulaer} gibt dazu eine Übersicht über die Abbildung der Aussagen.

\begin{table}
	\centering
	\begin{tabular}{|l|l|}
		\hline 
		Graphentyp aus Lemma \ref{lemma:lokal} \emph{A} & Definition aus Lemma \ref{lemma:lokal_regulaer} \\ 
		\hline 
		Leerer Graph & $d=0$ \\ 
		\hline 
		Vollständiger Graph & $d=n-1$ \\ 
		\hline 
		Matching Graph & $d=1$ \\ 
		\hline 
		Komplement eines Matching Graphen & $d=n-2$ \\ 
		\hline 
		5er Zyklus & $d=2$ und $n=5$ \\ 
		\hline 
	\end{tabular}
	\caption{Abbildung der Aussagen aus Lemma \ref{lemma:lokal} \emph{A} und Lemma \ref{lemma:lokal_regulaer}}
	\label{tab:mapping_regulaer}
\end{table}

\begin{Lemma}
	Sei $G$ ein bipartiter Graph mit den beiden Komponenten $X$ und $Y$. 
	Die Isomorphieeigenschaften von $G$ sind genau dann vollständig dadurch definiert, dass die $m$ Knoten aus $X$ den Grad $c$ und die $n$ Knoten aus $Y$ den Grad $d$ haben, wenn $c\in \{0,1,n-1,n\}$ oder $d\in \{0,1,m-1,m\}$ gilt.
	\label{lemma:lokal_bipartit}
\end{Lemma}

Bewiesen wird diese Aussage in \cite{koren1976pairs}.
Auch für bipartite Graphen können die Aussagen von Lemma \ref{lemma:lokal} (B) und Lemma \ref{lemma:lokal_bipartit} aufeinander abgebildet werden, was in Tabelle \ref{tab:mapping_bipartit} übersichtlich dargestellt wird. \\

\begin{table}
	\centering
	\begin{tabular}{|l|l|}
		\hline 
		Graphentyp aus Lemma \ref{lemma:lokal} \emph{B} & Definition aus Lemma \ref{lemma:lokal_bipartit} \\ 
		\hline 
		Leerer Graph & $d=c=0$ \\ 
		\hline 
		Vollständiger bipartiter Graph & $d=m$ und $c=n$ \\ 
		\hline 
		Disjunkte Vereinigung von Sternen & $d=1$ oder $c=1$ \\ 
		\hline 
		Komplement einer disjunkten Vereinigung von Sternen & $d=m-1$ oder $c=n-1$ \\ 
		\hline 
	\end{tabular}
	\caption{Abbildung der Aussagen aus Lemma \ref{lemma:lokal} \emph{B} und Lemma \ref{lemma:lokal_bipartit}}
	\label{tab:mapping_bipartit}
\end{table}

Wenn ein Graph $G$ nun einen Subgraphen $G[X]$ oder $G[X,Y]$ mit $X,Y\in \mathcal{P}_G$ enthält, dessen Typ allerdings nicht in Lemma \ref{lemma:lokal} aufgelistet wird, so lässt sich dieser durch einen nicht-isomorphen regulären oder biregulären Graphen mit gleichem Grad ersetzen. Diese Aussage folgt daraus, dass dessen Isomorphieeigenschaften wie in Lemma \ref{lemma:lokal_regulaer} und \ref{lemma:lokal_bipartit} dargestellt nicht allein durch ihre Parameter definiert sind, sodass das Ersetzen durch einen nicht-isomorphen Graphen mit eben diesen Parametern ermöglicht wird.

Diese Ergebnisse lassen sich für den Beweis von Lemma \ref{lemma:lokal} nutzen, wenn gezeigt wird, dass der durch das Ersetzen des Subgraphen entstandene Graph $H$ vom Color Refinement nicht von $G$ unterschieden werden kann. Im Folgenden gilt dazu, dass $G$ und $H$ die gleiche Knotenmenge enthalten und die Farben für einen Knoten $u$ durch $C^i_G(u)$ und $C^i_H(u)$ für die jeweiligen Graphen definiert sind.

\begin{Lemma}
	Seien $X$ und $Y$ Zellen der stabilen Partition eines Graphen $G$.
	\begin{enumerate}[label=(\alph*)]
		\item  Ist der Graph $H$ aus $G$ erstellt worden, indem die Kanten eines Subgraphen $G[X]$ mit denen eines regulären Graphen mit dem gleichen Grad auf der gleichen Knotenmenge $X$ ersetzt wurde, so gilt: $C^i_G(u)=C^i_H(u)$ für alle $u\in V(G)$ und beliebiges $i$.
		\item Ist der Graph $H$ aus $G$ erstellt worden, indem die Kanten eines Subgraphen $G[X,Y]$ mit denen eines biregulären Graphen mit dem gleichen Grad und den gleichen Partitionen ersetzt wurden, sodass die Knotengrade erhalten bleiben, gilt somit: $C^i_G(u)=C^i_H(u)$ für alle $u\in V(G)$ und beliebiges $i$.
	\end{enumerate}
	\label{lemma:lokal_nicht_unterscheidbar}
\end{Lemma}

\emph{Beweis von Lemma \ref{lemma:lokal}:}

\emph{(A)} Sei $G[X]$ ein Graph, dessen Struktur nicht in Lemma \ref{lemma:lokal} \emph{(A)} aufgelistet wird, so ist dieser wie in Lemma \ref{lemma:lokal_regulaer} dargestellt, kein Unigraph. Somit ist es möglich $G[X]$ innerhalb von $G$ durch einen nicht-isomorphen Graphen mit gleichen Parametern zu ersetzen. Der so entstandene Graph $H$ ist nicht isomorph zum Ursprungsgraphen $G$, da die Subgraphen $G[X]$ und $H[X]$ nicht isomorph sind. Teil (a) von Lemma \ref{lemma:lokal_nicht_unterscheidbar} hingegen zeigt, dass Gleichung \ref{eq:2} für beliebige $i$ erfüllt ist und das Color Refinement die beiden Graphen somit nicht unterscheiden kann. Aus dieser Aussage folgt, dass der Graph $G$ nicht zu der Klasse der CR-Graphen gehört. Durch Kontraposition folgt nun, dass ein CR-Graph die in Lemma \ref{lemma:lokal} \emph{(A)} genannten Eigenschaften in jedem Fall erfüllt.

\emph{(B)} Diese Aussage lässt sich unter Zuhilfenahme von Lemma \ref{lemma:lokal_bipartit} und Lemma \ref{lemma:lokal_nicht_unterscheidbar} (ii) sehr ähnlich zu \emph{(A)} begründen.$\hfill\square$


\section{Globale Struktur von CR-Graphen}
\label{sec/struktur_global}

%\begin{itemize}
%	\item Erklärung der globalen Struktur zugänglicher Graphen (C,D,E,F,G,H)
%	\item Ansonten wie Kapitel \ref{sec/struktur_lokal}
%\end{itemize}

Zusätzlich zur lokalen Struktur ist außerdem die globale Struktur von CR-Graphen interessant.
Der Begriff globale Struktur gibt hier an, dass der Zellgraph von $G$ bestimmte Eigenschaften erfüllen muss.

\begin{Definition}
	Der Zellgraph $C(G)$ eines Graphen $G$ wird aus dessen stabilen Partition $\mathcal{P}_G$ gebildet.
	Es handelt sich dabei um einen vollständigen Graphen, bei dem die Knoten die Zellen von $\mathcal{P}_G$ darstellen.
\end{Definition}

\subsection{Mögliche Eigenschaften von Zellgraphen}
Für das Verständnis der folgenden Eigenschaften von CR-Graphen ist es erforderlich einige Begriffe einzuführen und Eigenschaften von Zellgraphen zu erklären und zu benennen, wozu einige Definitionen folgen.
Die Knoten eines Zellgraphen, auch als Zellen bezeichnet, können dabei folgende Eigenschaften aufweisen.

\begin{Definition}
	Eine Zelle $X\in C(G)$ wird \emph{homogen} genannt, wenn der Graph $G[X]$ vollständig oder leer ist. Anderenfalls wird diese \emph{heterogen} genannt.
\end{Definition}

\begin{Definition}
	Für eine heterogene Zelle $X\in C(G)$ finden sich je nach Beschaffenheit von $G[X]$ die Bezeichnungen \emph{matching}, \emph{co-matching} oder \emph{pentagonal}.
	Eine homogene Zelle wird dagegen entweder \emph{leer} oder \emph{vollständig} genannt.
\end{Definition}

Für die Kanten eines Zellgraphen finden sich ebenfalls unterschiedliche Bezeichnungen, welche deren Beschaffenheit beschreiben.

\begin{Definition}
	Eine Kante ${X,Y}$ mit $X,Y\in C(G)$ wird \emph{isotrop} genannt, wenn der bipartite Graph $G[X,Y]$ vollständig oder leer ist. Anderenfalls wird diese \emph{anisotrop} genannt.
\end{Definition}

\begin{Definition}
	Eine anisotrope Kante ${X,Y}$ wird \emph{Konstellation} genannt, wenn $G[X,Y]$ eine disjunkte Vereinigung von Sternen ist.
	Anderenfalls wird diese \emph{Co-Konstellation} genannt.
	Bei Co-Konstellationen bildet das \glslink{bipartites_komplement}{bipartite Komplement} von $G[X,Y]$ eine disjunkte Vereinigung von Sternen.
	Eine isotrope Kante dagegen wird entweder \emph{leer} oder \emph{vollständig} genannt.
\end{Definition}

\begin{Definition}
	 Ein Pfad $X_1X_2...X_l$ in $C(G)$, bei dem jede Kante ${X_i,X_{i+1}}$ anisotrop ist, wird \emph{anisotroper Pfad} genannt. Wenn dieser Pfad einen Kreis schließt, wird er als \emph{anisotroper Zyklus} bezeichnet. Gilt für einen anisotropen Pfad $|X_1|=|X_2|=...=|X_l|$ dann wird er \emph{gleichmäßig} genannt.
\end{Definition}

\subsection{Allgemeine globale Eigenschaften von CR-Graphen}
Mit dem so gewonnenen Hintergrundwissen kann nun das folgende Lemma formuliert werden, welches globale Eigenschaften von CR-Graphen definiert.

\begin{Lemma}
	Der Zellgraph $C(G)$ eines CR-Graphen $G$ erfüllt folgende Eigenschaften:

	\begin{enumerate}[label=(\Alph*)]
		\setcounter{enumi}{2}
		\item $C(G)$ enthält keinen gleichmäßigen, anisotropen Pfad, der zwei heterogene Zellen verbindet
		\item $C(G)$ enthält keinen gleichmäßigen, anisotropen Zyklus
		\item $C(G)$ enthält weder einen anisotropen Pfad $XY_1Y_2...Y_lZ$, sodass $|X|<|Y_1|=|Y_2|=...=|Y_l|>|Z|$, noch einen anisotropen Zyklus $XY_1Y_2...Y_l$, sodass $|X|<|Y_1|=|Y_2|=...=|Y_l|$ und die Zelle $Y_l$ heterogen ist
		\item $C(G)$ enthält keinen anisotropen Pfad $XY_1Y_2...Y_l$, sodass $|X|<|Y_1|=|Y_2|=...=|Y_l|$ und die Zelle $Y_l$ heterogen ist
	\end{enumerate}
	\label{lemma:global1}
\end{Lemma}

\emph{Beweis C:} Angenommen $P$ wäre, entgegen Bedingung \emph{C}, ein gleichmäßiger, anisotroper Pfad in $C(G)$, welcher die beiden anisotropen Komponenten $X$ und $Y$ verbindet.
Da es sich um einen gleichmäßigen Pfad handelt, besitzen sämtliche Komponenten auf dem Pfad die Kardinalität $k=|X|=|Y|$.
Alle Kanten in $P$, die eine Co-Konstellation bilden, werden nun komplementiert, sodass diese eine Konstellation ergeben.
Da nun sämtliche Kanten aus $P$ eine Konstellation darstellen und die Kardinalität aller Komponenten identisch ist, können die Sterne, welche die disjunkte Vereinigung für die Konstellationen bilden nur eine einzige Form annehmen.
Diese bestehen aus einem Zentralknoten und einem Blatt, wobei diese in unterschiedlichen Komponenten liegen.
Dadurch, dass sämtliche Kanten in $P$ diese Form annehmen ergeben sich $k$ knotendisjunkte Pfade zwischen $X$ und $Y$.
Es sind also alle Knoten aus $X$ mit genau einem Knoten aus $Y$ durch einen Pfad verbunden und andersherum.
Sei $v\in X$ durch einen solchen Pfad mit seinem Gegenstück $v^*\in Y$ verbunden.
Der Pfad $P$ wird conducting genannt, wenn diese Abbildung von Knoten einen Isomorphismus von $G[X]$ auf $G[Y]$ darstellt.
Dazu müssen zwei Knoten $u,v\in X$ genau dann adjazent sein wenn ihre Gegenstücke $u^*,v^*\in Y$ adjazent sind.
Angenommen eine der beiden Komponenten ist matching und die andere ist co-matching, dann wird $P$ ebenfalls conducting genannt, wenn die Isomorphie zwischen der matching Komponente und dem Komplement der co-matching Komponente besteht.

Nachfolgend wird ein Graph $H$ erstellt werden, der nicht zu $G$ isomorph ist, welchen das Color Refinement allerdings nicht von $G$ unterscheiden kann.
Da $X$ und $Y$ heterogen sind, können die Kanten des Subgraphen $G[X]$ durch einen isomorphen aber unterschiedlichen Graphen mit der gleichen Knotenmenge $X$ ersetzt werden.
Ist $P$ in $G$ conducting, so muss $G[X]$ so ersetzt werden, dass der entstehende Pfad in $H$ nicht conducting ist, anderenfalls muss der entsprechende Pfad in $H$ conducting sein.
Ein Beispiel für ein solches Ersetzen findet sich in Abbildung \ref{fig:global_c}.\\

\begin{figure}[t]
	\centering
	\fbox{\includegraphics[width=0.75\textwidth]{./img/global_c.jpg}}
	\caption{Beispiel für das Ersetzen eines Subgraphen in Lemma \ref{lemma:global1} \emph{C}}
	\label{fig:global_c}
\end{figure}

Lemma \ref{lemma:lokal_nicht_unterscheidbar} (a) besagt nun, dass der Color Refinement Algorithmus für $G$ und $H$ die gleiche Färbung errechnet und somit nicht zwischen den beiden unterscheiden kann.
Außerdem besagt Lemma \ref{lemma:faerbung_isomorphismus}, dass jeder Isomorphismus zwischen $G$ und $H$ Zellen auf sich selbst abbilden muss.
Da also gilt $\phi (v^*)=\phi (v)^*$, muss $\phi $ die Eigenschaft von $P$ erhalten conducting oder nicht zu sein, was in diesem konstruierten Beispiel nicht erfüllt ist.
Es folgt, dass $G$ und $H$ nicht isomorph sind und $G$ daher kein CR-Graph sein kann.

\emph{Beweis D:} Es wird wieder entgegen der Bedingung \emph{D} angenommen, dass $C(G)$ einen gleichmäßigen, anisotropen Zyklus $Q$ der Länge $m$ enthält.
Da dieser gleichmäßig ist, besitzen sämtliche Zellen in $Q$ die gleiche Kardinalität $k$.
Wie im Beweis von \emph{C} kann nun der Graph $G[A,B]$ für jede Co-Konstellation ${A,B}$ komplementiert werden, sodass ein Zyklus entsteht, welcher nur Konstellationen als Kanten enthält.
Ebenfalls wie im Beweis von \emph{C} gibt es eine eins zu eins Beziehung zwischen den Knoten zweier Benachbarter Komponenten, wodurch eine knotendisjunkte Vereinigung von Kreisen in $G$ entsteht.
Alle Kreise besitzen dabei ein Vielfaches der Länge $m$, wobei so im Extremfall ein Kreis der Länge $km$ oder $k$ Kreise der Länge $m$ vorliegen.

Durch die Anzahl der Kreise und deren Länge können die Isomorphieeigenschaften des entstehenden Subgraphen vollständig beschreiben, sodass der Isomorphietyp für den Zyklus $Q$ nachfolgend als $\tau (Q)$ bezeichnet wird.
Sei $\phi $ ein Isomorphismus von $G$ zu einem anderen Graphen $H$, so ist $\phi '$ der auf den Zellgraphen übertragene Isomorphismus von $C(G)$ zu $C(H)$.
Durch die Definition von $\tau $ ergibt sich, dass $\tau (\phi '(Q))=\tau (Q)$ gelten muss.
Intuitiv bedeutet dies, dass ein Kreis einer bestimmten Größe nur auf einen anderen Kreis abgebildet werden kann, welcher die selbe Größe besitzt.

Seien $X,Y\in Q$ zwei aufeinanderfolgende Zellen, so kann der dadurch entstehende Subgraph $G[X,Y]$ durch einen isomorphen aber nicht identischen Graphen ersetzt werden, da die Kante {X,Y} anisotrop ist.
Somit entsteht ein neuer Graph $H$, welcher beispielsweise so konstruiert sein kann, dass für den betrachteten Zyklus die beiden eingangs erwähnten Extreme vorliegen.
Dadurch, dass sich hier zwei Möglichkeiten ergeben wie $\tau (Q)$ nach dem Austauschen aussehen kann, ist es möglich den Subgraphen $G[X,Y]$ so zu ersetzen, dass $\tau (Q)$ sich ändert.

Es wird wieder das Lemma \ref{lemma:lokal_nicht_unterscheidbar} genutzt, um zu zeigen, dass das Color Refinement nicht zwischen $G$ und $H$ unterschieden werden kann.
Außerdem ist $G$ nicht isomorph zu $H$, da sich $H$ so konstruieren lässt, dass $\tau (Q)$ in beiden Graphen unterschiedlich ist, wodurch der Graph $G$ nicht zu der Klasse der CR-Graphen gehören kann.

\subsection{Baumartige Struktur von CR-Graphen}
Bei genauerer Betrachtung fällt auf, dass der Zellgraph von CR-Graphen eine baumartige Struktur aufweist.
Um das folgende Lemma verstehen zu können, wird der Begriff \emph{anisotrope Komponente} benötigt.
\begin{Definition}
	In einem Zellgraphen $C(G)$ bezeichnet eine \emph{anisotrope Komponente} einen Subgraphen, dessen Kanten alle isotrop sind.
	Wenn eine Zelle keine inzidente, anisotrope Kante besitzt, dann ergibt sich daraus eine anisotrope Komponente mit nur einer Zelle.
\end{Definition}

\begin{Lemma}
	Angenommen ein CR-Graph $G$ erfüllt die Bedingungen \emph{A-F} aus den Lemmata \ref{lemma:lokal} und \ref{lemma:global1}.
	Für jede anisotrope Komponente $A$ von $C(G)$ gelten folgende Eigenschaften:
	
	\begin{enumerate}[label=(\Alph*)]
		\setcounter{enumi}{6}
		\item $A$ ist ein Baum, der folgende Monotonieeigenschaft erfüllt: Sei $R$ eine Zelle aus $A$ mit minimaler Kardinalität, so ist $A_R$ der gerichtete Baum mit Wurzel $R$; Für jede gerichtete Kante $(X,Y)$ aus $A_R$ gilt dann $|X|\leq |Y|$
		\item $A$ enthält maximal eine heterogene Zelle; Wenn eine solche Zelle existiert, hat diese minimale Kardinalität in $A$
	\end{enumerate}
	\label{lemma:global2}
\end{Lemma}



\section{Ergebnis und Laufzeit}
\label{sec/ergebnis}

%\begin{itemize}
%	\item Erklärung von Theorem 9
%	\item Vorstellung des Ergebnisses
%	\item Beweis der Laufzeit von $O((n+m)\log n)$
%\end{itemize}

Die vorgestellten lokalen und globalen Eigenschaften von CR-Graphen reichen wie im Folgenden gezeigt aus, um hinreichende Bedingungen für CR-Graphen zu formulieren und darauf basierend ein effizientes Verfahren für das Erkennen solcher Graphen zu entwickeln.

\subsection{Hinreichende Bedingungen für das Erkennen von CR-Graphen}
\begin{Theorem}
	Für einen Graphen $G$ sind folgende Aussagen äquivalent:
	
	\begin{enumerate}[label=(\alph*)]
		\item $G$ ist ein CR-Graph
		\item $G$ erfüllt Bedingungen \emph{A-F}
		\item $G$ erfüllt Bedingungen \emph{A}, \emph{B}, \emph{G} und \emph{H}
	\end{enumerate}
\end{Theorem}

\emph{Beweis:} Die Äquivalenz der Aussagen wird gezeigt, indem gezeigt wird, dass gilt: $(a)\rightarrow (b)\rightarrow (c)\rightarrow (a)$.
Die bisher erlangten Erkenntnisse ermöglichen es bereits einen großen Teil dieser Aussage zu bestätigen. Somit wurde in den Lemmata \ref{lemma:lokal} und \ref{lemma:global1} gezeigt, dass $(a)\rightarrow (b)$ gilt. Ebenfalls wurde in Lemma \ref{lemma:global2} gezeigt, dass $(b)\rightarrow (c)$ gilt. Es bleibt also nur noch zu zeigen, dass auch $(c)\rightarrow (a)$ gültig ist.

[Beweis folgt]

\subsection{Laufzeit}
Zur Berechnung der Laufzeit wird im Folgenden davon ausgegangen, dass der Graph $G$ in Adjazenzlistendarstellung vorliegt.
Nach \cite{CARDON198285} lässt sich die stabile Partition eines Graphen $G$ in Zeit $\mathcal{O}((n+m)\log n)$ berechnen.

\begin{Theorem}
	Die Klasse der CR-Graphen ist in Zeit $\mathcal{O}((n+m)\log n)$ entscheidbar. Dabei bezeichnet $n$ die Anzahl der Knoten und $m$ die Anzahl der Kanten des Eingabegraphen.
\end{Theorem}

\emph{Beweis:} Zunächst wird die stabile Partition $\mathcal{P}_G=\{X_1,X_2,...,X_k\}$ berechnet, was wie eingangs erwähnt die Laufzeit $\mathcal{O}((n+m)\log n)$ benötigt.
Außerdem wird $C^*(G)$ definiert als der Zellgraph, bei dem sämtliche leeren Kanten, also solche bei denen keine Verbindungen zwischen den Elementen der beiden Endpunkte besteht, entfernt wurden.

Für die Elemente $X_i\in C^*(G)$ werden die Adjazenzlisten gebildet, indem die Adjazenzliste eines beliebigen Knoten $u\in X_i$ durchlaufen wird und sämtliche Zellen aufgelistet werden, welche einen zu $u$ adjazenten Knoten enthalten.
Die dadurch gewonnenen Informationen sind identisch für alle Knoten aus $X_i$, da diese alle gleichartige Nachbarschaften besitzen, weshalb es ausreicht die Operation für einen beliebigen Knoten durchzuführen.
Durch die Informationen aus der Adjazenzliste lässt sich der Grad der Knoten innerhalb der Zellen bestimmen und somit leicht Bedingung \emph{A} durch die in Lemma \ref{lemma:lokal_regulaer} vorgestellten Bedingungen für jeden Subgraphen $G[X_i]$ überprüfen.

Für jede Kante ${X_i,X_j}$ aus $C^*(G)$ wird der Wert $d_{ij}$ berechnet, welcher die Anzahl der Nachbarn in $X_j$ beschreibt, zu denen jeder Knoten aus $X_i$ adjazent ist.
Dieser Wert wird ebenfalls für den Fall $i=j$ betrachtet, wobei die Nachbarn innerhalb der Zelle gezählt werden.
Dadurch, dass die Werte $|X_i|$, $|X_j|$ und $d_{ij}$ nun bekannt sind, lässt sich Bedingung \emph{B} durch die in Lemma \ref{lemma:lokal_bipartit} vorgestellten Bedingungen überprüfen.

Da alle Zellen des Zellgraphen $C^*(G)$ im schlimmsten Falle unterschiedliche Farben haben, gibt es maximal $n$ Zellen.
Mit dem beschriebenen Verfahren können die Bedingungen \emph{A} und \emph{B} somit in der Zeit $\mathcal{O}(n)$ überprüft werden.

Zum Überprüfen von Bedingung \emph{H} wird eine Breitensuche auf dem Zellgraphen $C^*(G)$ durchgeführt, welche alle anisotropen Komponenten findet und gleichzeitig überprüft, ob diese eine Baumstruktur aufweisen und nur eine anisotrope Komponente enthält.

Wird diese Breitensuche nun von der Zelle minimaler Kardinalität für jede Komponente wiederholt, so lässt sich die in Bedingung \emph{G} beschriebene Monotonieeigenschaft für jede Kante überprüfen.

Der Breitensuchealgorithmus benötigt eine Laufzeit von $\mathcal{O}(n+m)$, da der Zellgraph maximal $n$ Knoten und maximal $m$ Kanten besitzt.

Es ist somit zu erkennen, dass das Errechnen der stabilen Partition, auf der die ganzen Operationen ausgeführt werden mit $\mathcal{O}((n+m)\log n)$ die dominierende Laufzeit des Algorithmus ist.

\section{Fazit}

\begin{itemize}
	\item Zusammenfassung der Ergebnisse
	\item Bezugnehmen auf die Einleitung und ob die Ziele erreicht wurden
	\item Abschluss der Arbeit
\end{itemize}


%\subsection{Beispiele für Referenzen}
%
%\begin{itemize}
% \item Die starken Zusammenhangskomponenten eines gerichteten Graphen können in Linearzeit bestimmt werden~\citep{Tar72}.
% \item \citet{Tar72} hat gezeigt, dass \dots
%\end{itemize}
%
%
%\subsection{Theorem-Umgebung}
%
%% Vordefinierte Umgebungen:
%% Theorem, Lemma, Proposition, Satz, Korollar, Definition, Beispiel, Anmerkung, Bemerkung, Beweis
%\begin{Theorem}[Optionaler Titel]
% Aussage.
%\end{Theorem}


%------------------------------------------------
% Literaturverzeichnis
%------------------------------------------------

\bibliographystyle{abbrvnat-ger}
\bibliography{literatur}
\addcontentsline{toc}{section}{\bibname}


\end{document}

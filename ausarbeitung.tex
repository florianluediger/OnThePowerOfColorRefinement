%------------------------------------------------
% Latex-Grundger�st f�r Seminarausarbeitungen
%
% zu Erzeugen mit
%   pdflatex ausarbeitung.tex
%
% von Carsten Gutwenger, Nils Kriege
% 
%------------------------------------------------

\documentclass[a4paper,12pt,twoside]{article}
\usepackage[a4paper,left=3.5cm,right=2.5cm,bottom=3.5cm,top=3cm]{geometry}

% Encoding der .tex-Datei
% muss je nach Editor-Einstellungen eventuell angepasst werden!
%   latin1    Unix/Windows
%   ansinew   Alternative f�r Windows
%   applemac  Apple
%   utf8      Alternative f�r Unix/Windows/Apple
\usepackage[latin1]{inputenc}

\usepackage[ngerman]{babel}                                 % deutsche Sprache
\usepackage[T1]{fontenc}                                    % Unterst�tzung f�r Umlaute mit Fonts
\usepackage[pdftex]{graphicx,color}                         % Einbetten von Grafiken, Farbe
\usepackage[format=plain,small,bf,indention=7.5mm]{caption} % verbesserte Beschriftung von Abbildungen
\usepackage{subfig}                                         % subfigures
\usepackage{url}                                            % \url{}-Kommando
\usepackage{fancyhdr,float}                                 % Kopf-/Fu�zeilen
\usepackage[pdftex,pdfpagelabels]{hyperref}                 % Hyperlinks im PDF
\usepackage[round]{natbib}                                  % Literaturverzeichnis
\usepackage{doi}                                            % DOIs
\usepackage[section,boxed]{algorithm}                       % Floating-Umgebung f�r Algorithmen
\usepackage{algpseudocode}                                  % Pseudocode f�r Algorithmen
\usepackage[standard]{ntheorem}                             % Theorem-Umgebung

\captionsetup{farskip=10pt,topadjust=0pt,captionskip=10pt,nearskip=0pt,margin=10pt}

\pagestyle{fancy}
\fancyhead[LE,RO]{\thepage}
\fancyhead[RE]{\nouppercase{\slshape \leftmark}}
\fancyhead[LO]{\nouppercase{\slshape \rightmark}}
\fancyfoot[C]{}


\begin{document}

%------------------------------------------------
% Titelseite
%------------------------------------------------

\begin{titlepage}
\vspace*{-2cm}
\newlength{\links}
\setlength{\links}{-1.5cm} \sf \LARGE

\hspace*{\links}
\begin{minipage}{12.5cm}
\includegraphics[width=8cm]{tud_logo_rgb}
\end{minipage}

\vspace*{4cm}

\large
\begin{center}
{\Large Seminarausarbeitung} \\[1ex]
{\LARGE\textbf{Titel der Ausarbeitung}}\\[3ex]
Name des Seminarteilnehmers\\[1ex]
\today\\[7ex]
im Rahmen des Seminars\\[1ex]
{\Large\textbf{Algorithm Engineering}}\\[1ex]
von Prof.~Dr.~Petra Mutzel\\[1ex]
Wintersemester 2012/13
\end{center}

\vspace*{4cm}
\hspace*{\links}
\begin{minipage}[b]{15cm}
\normalsize \raggedright
\textbf{Betreuer:} \\
Name des Betreuers \\[2ex]

\textbf{Basierend auf:}\\
Autoren, Titel des Papers, Quelle
\end{minipage}

\definecolor{TUGreen}{rgb}{0.517,0.721,0.094}
\vfill
\hspace*{\links}
\begin{minipage}[b]{8cm}
\normalsize \raggedright
Fakult�t f�r Informatik\\
Algorithm Engineering (Ls11)\\
Technische Universit�t Dortmund
\end{minipage}

\end{titlepage}

%------------------------------------------------
% Inhaltsverzeichnis
%------------------------------------------------

\tableofcontents
\clearpage


%------------------------------------------------
% Einf�hrung
%------------------------------------------------

\section{Einf�hrung}

Lorem ipsum dolor sit amet, consetetur sadipscing elitr, sed diam nonumy eirmod tempor invidunt ut labore et dolore
magna aliquyam erat, sed diam voluptua. At vero eos et accusam et justo duo dolores et ea rebum. Stet clita kasd
gubergren, no sea takimata sanctus est Lorem ipsum dolor sit amet. Lorem ipsum dolor sit amet, consetetur sadipscing
elitr, sed diam nonumy eirmod tempor invidunt ut labore et dolore magna aliquyam erat, sed diam voluptua. At vero eos
et accusam et justo duo dolores et ea rebum. Stet clita kasd gubergren, no sea takimata sanctus est Lorem ipsum dolor
sit amet.


\subsection{Beispiele f�r Referenzen}

\begin{itemize}
 \item Die starken Zusammenhangskomponenten eines gerichteten Graphen k�nnen in Linearzeit bestimmt werden~\citep{Tar72}.
 \item \citet{Tar72} hat gezeigt, dass \dots
\end{itemize}


\subsection{Theorem-Umgebung}

% Vordefinierte Umgebungen:
% Theorem, Lemma, Proposition, Satz, Korollar, Definition, Beispiel, Anmerkung, Bemerkung, Beweis
\begin{Theorem}[Optionaler Titel]
 Aussage.
\end{Theorem}


%------------------------------------------------
% Literaturverzeichnis
%------------------------------------------------

\bibliographystyle{abbrvnat-ger}
\bibliography{literatur}
\addcontentsline{toc}{section}{\bibname}


\end{document}

\documentclass{beamer}
\usetheme{metropolis}

\usepackage[utf8]{inputenc}
\usepackage[ngerman]{babel}                                 % deutsche Sprache
\usepackage[T1]{fontenc}                                    % Unterstützung für Umlaute mit Fonts
\usepackage[autostyle=true,german=quotes]{csquotes}			% Wird für den Befehl \enquote benötigt
\usepackage{tikz}											% Ermöglicht das Zeichnen von Grafiken
\usepackage{amssymb}										% Einfügen von zusätzlichen Symbolen wie \square
\usepackage{fontspec}
\usepackage{enumitem}										% Erweiterte Funktionen der enumerate Umgebung
\usepackage{marvosym}										% Einfügen nützlicher Symbole

% Einstellen von Formatierungen
\setsansfont{Calibri}
\setmonofont{Courier New}

\metroset{titleformat=smallcaps,		% Überschriften werden in smallcaps geschrieben
	sectionpage=simple,					% Entfernen der Progressbar auf den Sectionpages
	block=fill,							% Hinterlegen von Umgebungen wie Theorem oder Example
}							

% Treffen von Einstellungen für tikz
\usetikzlibrary{positioning}
\usetikzlibrary{fit}
\usetikzlibrary{shapes.geometric}
\tikzset{main node/.style={circle,fill=blue!50,draw,minimum size=0.7cm,inner sep=0pt},}
\tikzset{bezier/.style={inner sep=0pt,minimum size=0cm},}

% Erstellen eines Commands zum verringern der Seitenränder für eine Folie
\newcommand\Wider[2][3em]{%
	\makebox[\linewidth][c]{%
		\begin{minipage}{\dimexpr\textwidth+#1\relax}
			\raggedright#2
		\end{minipage}%
	}%
}

% Erstellen eines Commands zum Zeichnen von Zahlen mit einem Kreis darum
\newcommand\circlearound[1]{%
	\tikz[baseline]\node[draw,shape=circle,anchor=base] {#1} ;}

\newtheorem{Korollar}{Korollar}

\title{On the Power of Color Refinement}
\subtitle{V. Arvind, Johannes Köbler, Gaurav Rattan und\\ Oleg Verbitsky}
\date{05.02.2018}
\author{Florian Lüdiger}
\institute{Seminar Algorithm Engineering - Lehrstuhl 11 - TU Dortmund}
\begin{document}
	\maketitle
	
	\section{Wiederholung: Graph-Isomorphie und Color-Refinement}
	\begin{frame}{Graph-Isomorphie}
		\begin{Definition}
			 Zwei Graphen $G$ und $H$ sind \alert{\textbf{isomorph}}, wenn es eine bijektive Abbildung $\phi$ gibt, sodass gilt:\\ $(u,v)\in E_G\Leftrightarrow (\phi (u),\phi (v))\in E_H$ für alle $u,v\in V_G$.
		\end{Definition}
		% Vortragsnotizen:
		% Adjazenz
		% Isomorphismus
		% kein Polynomialzeitalgorithmus bekannt
	\end{frame}
	\begin{frame}{Beispiel}
		\centering
		\begin{tikzpicture}
			\begin{scope}
				\node[draw=none,minimum size=4cm,regular polygon,regular polygon sides=5] (a) {};
				\foreach \x in {1,2,...,5}
				\node[main node] at (a.corner \x) (\x) {};
				
				\path[draw,thick]
				(1) edge node {} (2)
				(2) edge node {} (3)
				(3) edge node {} (4)
				(4) edge node {} (5)
				(5) edge node {} (1);
			\end{scope}
			
			\begin{scope}[xshift=6cm]
				\node[draw=none,minimum size=4cm,regular polygon,regular polygon sides=5] (a) {};
				\foreach \x in {1,2,...,5}
					\node[main node] at (a.corner \x) (\x) {};
				
				\path[draw,thick]
				(1) edge node {} (4)
				(4) edge node {} (2)
				(2) edge node {} (5)
				(5) edge node {} (3)
				(3) edge node {} (1);
			\end{scope}
		\end{tikzpicture}
	\end{frame}
	\begin{frame}{Beispiel}
		\centering
		\begin{tikzpicture}
			\begin{scope}
				\node[draw=none,minimum size=4cm,regular polygon,regular polygon sides=5] (a) {};
				\foreach \x in {1,2,...,5}
					\node[main node] at (a.corner \x) (\x) {\x};
				
				\path[draw,thick]
				(1) edge node {} (2)
				(2) edge node {} (3)
				(3) edge node {} (4)
				(4) edge node {} (5)
				(5) edge node {} (1);
			\end{scope}
			
			\begin{scope}[xshift=6cm]
				\node[draw=none,minimum size=4cm,regular polygon,regular polygon sides=5] (a) {};
				\foreach \x in {1,2,...,5}
					\node[main node] at (a.corner \x) (\x) {\x};
				\node[main node] at (a.corner 1) (1) {1};
				\node[main node] at (a.corner 2) (2) {4};
				\node[main node] at (a.corner 3) (3) {2};
				\node[main node] at (a.corner 4) (4) {5};
				\node[main node] at (a.corner 5) (5) {3};
				
				\path[draw,thick]
				(1) edge node {} (4)
				(4) edge node {} (2)
				(2) edge node {} (5)
				(5) edge node {} (3)
				(3) edge node {} (1);
			\end{scope}
		\end{tikzpicture}
	\end{frame}
	\begin{frame}{Color-Refinement}
		\begin{Definition}
			Mit der Color-Refinement-Heuristik kann in polynomieller Zeit festgestellt werden, dass zwei Graphen \alert{\textbf{nicht isomorph}} sind.
		\end{Definition}
		Anders gesagt gilt für beliebige Graphen $G$,$H$:\\
		\circlearound{1} CR unterscheidet $G$ und $H$ $\Rightarrow$ $G\not\simeq H$
	\end{frame}
	\begin{frame}{Beispiel}
		\Wider[4em]{
			\begin{tikzpicture}
		% BILD 1:
				\begin{scope}
					\node[main node] (1) {};
					\node[main node] (2) [right = 1cm of 1] {};
					\node[main node] (3) [below = 1cm of 1] {};
					\node[main node] (4) [right = 1cm of 3] {};
					\node[main node] (5) [below right = 0.35cm and 0.35cm of 1] {};
					\node (text) [left = 0cm of 1] {1)};
					
					\path[draw,thick]
					(1) edge node {} (2)
					(1) edge node {} (3)
					(3) edge node {} (4)
					(2) edge node {} (4)
					(1) edge node {} (5);
				\end{scope}
				
				\begin{scope}[xshift=3cm]
					\node[main node] (1) {};
					\node[main node] (2) [right = 1cm of 1] {};
					\node[main node] (3) [below = 1cm of 1] {};
					\node[main node] (4) [right = 1cm of 3] {};
					\node[main node] (5) [below right = 0.35cm and 0.35cm of 1] {};
					
					\path[draw,thick]
					(1) edge node {} (3)
					(3) edge node {} (5)
					(2) edge node {} (5)
					(2) edge node {} (4)
					(4) edge node {} (5);
				\end{scope}
				\pause
		% BILD 2:
				\begin{scope}[yshift=-3cm]
					\node[main node] (1) [fill=red!50] {};
					\node[main node] (2) [right = 1cm of 1] {};
					\node[main node] (3) [below = 1cm of 1] {};
					\node[main node] (4) [right = 1cm of 3] {};
					\node[main node] (5) [fill=green!50,below right = 0.35cm and 0.35cm of 1] {};
					\node (text) [left = 0cm of 1] {2)};
					
					\path[draw,thick]
					(1) edge node {} (2)
					(1) edge node {} (3)
					(3) edge node {} (4)
					(2) edge node {} (4)
					(1) edge node {} (5);
				\end{scope}
				
				\begin{scope}[yshift=-3cm,xshift=3cm]
					\node[main node] (1) [fill=green!50]{};
					\node[main node] (2) [right = 1cm of 1] {};
					\node[main node] (3) [below = 1cm of 1] {};
					\node[main node] (4) [right = 1cm of 3] {};
					\node[main node] (5) [fill=red!50,below right = 0.35cm and 0.35cm of 1] {};
					
					\path[draw,thick]
					(1) edge node {} (3)
					(3) edge node {} (5)
					(2) edge node {} (5)
					(2) edge node {} (4)
					(4) edge node {} (5);
				\end{scope}
				\pause
		% BILD 3:
				\begin{scope}[xshift=6.2cm]
					\node[main node] (1) [fill=red!50] {};
					\node[main node] (2) [right = 1cm of 1] {};
					\node[main node] (3) [below = 1cm of 1] {};
					\node[main node] (4) [fill=yellow!50,right = 1cm of 3] {};
					\node[main node] (5) [fill=green!50,below right = 0.35cm and 0.35cm of 1] {};
					\node (text) [left = 0cm of 1] {3)};
					
					\path[draw,thick]
					(1) edge node {} (2)
					(1) edge node {} (3)
					(3) edge node {} (4)
					(2) edge node {} (4)
					(1) edge node {} (5);
				\end{scope}
				
				\begin{scope}[xshift=9.2cm]
					\node[main node] (1) [fill=orange!50]{};
					\node[main node] (2) [right = 1cm of 1] {};
					\node[main node] (3) [fill=pink!50,below = 1cm of 1] {};
					\node[main node] (4) [right = 1cm of 3] {};
					\node[main node] (5) [fill=black!50,below right = 0.35cm and 0.35cm of 1] {};
					
					\path[draw,thick]
					(1) edge node {} (3)
					(3) edge node {} (5)
					(2) edge node {} (5)
					(2) edge node {} (4)
					(4) edge node {} (5);
				\end{scope}
				\pause
		% BILD 4:
				\begin{scope}[yshift=-3cm,xshift=6.2cm]
					\node[main node] (1) [fill=red!50] {};
					\node[main node] (2) [right = 1cm of 1] {};
					\node[main node] (3) [below = 1cm of 1] {};
					\node[main node] (4) [fill=yellow!50,right = 1cm of 3] {};
					\node[main node] (5) [fill=green!50,below right = 0.35cm and 0.35cm of 1] {};
					\node (text) [left = 0cm of 1] {4)};
					
					\path[draw,thick]
					(1) edge node {} (2)
					(1) edge node {} (3)
					(3) edge node {} (4)
					(2) edge node {} (4)
					(1) edge node {} (5);
				\end{scope}
				
				\begin{scope}[yshift=-3cm,xshift=9.2cm]
					\node[main node] (1) [fill=orange!50]{};
					\node[main node] (2) [fill=white!50,right = 1cm of 1] {};
					\node[main node] (3) [fill=pink!50,below = 1cm of 1] {};
					\node[main node] (4) [fill=white!50,right = 1cm of 3] {};
					\node[main node] (5) [fill=black!50,below right = 0.35cm and 0.35cm of 1] {};
					
					\path[draw,thick]
					(1) edge node {} (3)
					(3) edge node {} (5)
					(2) edge node {} (5)
					(2) edge node {} (4)
					(4) edge node {} (5);
				\end{scope}
			\end{tikzpicture}
		}
	\end{frame}
	\begin{frame}{Limitierung der Heuristik}
		Es gibt nicht-isomorphe Graphenpaare, welche das Color-Refinement nicht unterscheiden kann.
		
	
		\begin{tikzpicture}
			\begin{scope}
				\node[main node] (1) {};
				\node[main node] (2) [below left = 0.8cm and 1.4cm of 1]  {};
				\node[main node] (6) [below right = 0.8cm and 1.4cm of 1] {};
				\node[main node] (3) [below = 1.4cm of 2] {};
				\node[main node] (5) [below = 1.4cm of 6] {};
				\node[main node] (4) [below right = 0.8cm and 1.4cm of 3] {};
				
				\path[draw,thick]
				(1) edge node {} (2)
				(2) edge node {} (3)
				(3) edge node {} (4)
				(4) edge node {} (5)
				(5) edge node {} (6)
				(6) edge node {} (1);
			\end{scope}
			
			\node [below right = 0.4cm and 0.8cm of 6] {\huge $\not\simeq$};
			
			\begin{scope}[xshift=6.5cm]
				\node[main node] (1) {};
				\node[main node] (2) [below left = 0.8cm and 1.4cm of 1]  {};
				\node[main node] (6) [below right = 0.8cm and 1.4cm of 1] {};
				\node[main node] (3) [below = 1.4cm of 2] {};
				\node[main node] (5) [below = 1.4cm of 6] {};
				\node[main node] (4) [below right = 0.8cm and 1.4cm of 3] {};
				
				\path[draw,thick]
				(1) edge node {} (2)
				(2) edge node {} (6)
				(3) edge node {} (4)
				(4) edge node {} (5)
				(5) edge node {} (3)
				(6) edge node {} (1);
			\end{scope}
		\end{tikzpicture}
	\end{frame}

	\section{Was gibt es neues?}
	\begin{frame}{Die Klasse der CR-Graphen}
		\begin{Definition}
			Graph $G$ ist \alert{\textbf{CR-Graph}}, wenn das Color-Refinement diesen von jedem nicht zu $G$ isomorphen Graphen $H$ unterscheiden kann.
		\end{Definition}
		Für beliebige CR-Graphen $G$,$H$ gilt also:\\
		\circlearound{2} $G\not\simeq H$ $\Rightarrow$ CR unterscheidet $G$ und $H$
	\end{frame}
	\begin{frame}{Ergebnis und Beobachtung}
		\circlearound{1} CR unterscheidet $G$ und $H$ $\Rightarrow$ $G\not\simeq H$\\
		\circlearound{2} $G\not\simeq H$ $\Rightarrow$ CR unterscheidet $G$ und $H$
		\begin{Korollar}
			Für zwei CR-Graphen $G$ und $H$ gilt:\\
			CR erkennt $G$ und $H$ als isomorph $\Leftrightarrow$ $G\simeq H$
		\end{Korollar}
		\pause
		\vspace{10mm}
		\alert{\Large{\textit{Wie identifiziere ich also die Klasse der CR-Graphen?}}}
	\end{frame}

	\section{Begriffserklärungen und Anwendungsbeispiel}
	\begin{frame}{Anwendungsbeispiel}
		\centering
		\begin{tikzpicture}
			\def\scale{0.6}
			\begin{scope}
				\node[main node] (1) [fill=green!50]{A};
				\node[main node] (2) [fill=red!50,below left = \scale*0.57cm and \scale*1cm of 1]  {B};
				\node[main node] (6) [fill=red!50,below right = \scale*0.57cm and \scale*1cm of 1] {F};
				\node[main node] (3) [fill=red!50,below = \scale*1cm of 2] {C};
				\node[main node] (5) [fill=red!50,below = \scale*1cm of 6] {E};
				\node[main node] (4) [fill=green!50,below right = \scale*0.57cm and \scale*1cm of 3] {D};
				\node[main node] (7) [above = \scale*1cm of 1] {G};
				\node[main node] (8) [below = \scale*1cm of 4] {H};
				
				\path[draw,thick]
				(1) edge node {} (2)
				(2) edge node {} (3)
				(3) edge node {} (4)
				(4) edge node {} (5)
				(5) edge node {} (6)
				(6) edge node {} (1)
				(1) edge node {} (7)
				(4) edge node {} (8);
			\end{scope}
			
%			\begin{scope}[xshift=5cm]
%				\node[main node] (1) [fill=green!50]{A};
%				\node[main node] (2) [fill=red!50,below left = \scale*0.57cm and \scale*1cm of 1]  {B};
%				\node[main node] (6) [fill=red!50,below right = \scale*0.57cm and \scale*1cm of 1] {F};
%				\node[main node] (3) [fill=red!50,below = \scale*1cm of 2] {C};
%				\node[main node] (5) [fill=red!50,below = \scale*1cm of 6] {E};
%				\node[main node] (4) [fill=green!50,below right = \scale*0.57cm and \scale*1cm of 3] {D};
%				\node[main node] (7) [above = \scale*1cm of 1] {G};
%				\node[main node] (8) [below = \scale*1cm of 4] {H};
%				
%				\path[draw,thick]
%				(1) edge node {} (2)
%				(2) edge node {} (6)
%				(3) edge node {} (4)
%				(4) edge node {} (5)
%				(5) edge node {} (3)
%				(6) edge node {} (1)
%				(1) edge node {} (7)
%				(4) edge node {} (8);
%			\end{scope}
		\end{tikzpicture}
	\end{frame}
	\begin{frame}{Stabile Partitionierung}
		\begin{Definition}
			Die \alert{\textbf{Partitionierung}} $\mathcal{P}$ teilt den Graphen $G$ in die Farbklassen eines Verfeinerungsschritts ein.
		\end{Definition}
		\pause
		\begin{Definition}
			Wenn sich die Partitionierung bei weiteren Verfeinerungsschritten nicht mehr ändert, wird diese \alert{\textbf{stabile Partitionierung}} $\mathcal{P}^s$ genannt.
		\end{Definition}
		\pause
		\begin{Definition}
			Die einzelnen Partitionen innerhalb der Partitionierung werden \alert{\textbf{Zellen}} genannt.
		\end{Definition}
	\end{frame}
	\begin{frame}{Anwendung auf das Beispiel}
		\centering
		\begin{tikzpicture}
			\def\scale{0.6}
			\begin{scope}
				\node[main node] (1) [fill=green!50]{A};
				\node[main node] (2) [fill=red!50,below left = \scale*0.57cm and \scale*1cm of 1]  {B};
				\node[main node] (6) [fill=red!50,below right = \scale*0.57cm and \scale*1cm of 1] {F};
				\node[main node] (3) [fill=red!50,below = \scale*1cm of 2] {C};
				\node[main node] (5) [fill=red!50,below = \scale*1cm of 6] {E};
				\node[main node] (4) [fill=green!50,below right = \scale*0.57cm and \scale*1cm of 3] {D};
				\node[main node] (7) [above = \scale*1cm of 1] {G};
				\node[main node] (8) [below = \scale*1cm of 4] {H};
				
				\path[draw,thick]
				(1) edge node {} (2)
				(2) edge node {} (3)
				(3) edge node {} (4)
				(4) edge node {} (5)
				(5) edge node {} (6)
				(6) edge node {} (1)
				(1) edge node {} (7)
				(4) edge node {} (8);
			\end{scope}
			
			\node (B) [above right = -0.4cm and 2cm of 6] {$\rightarrow \mathcal{B}$};
			\node (G) [below = 0.2cm of B] {$\rightarrow \mathcal{G}$};
			\node (R) [below = 0.2cm of G] {$\rightarrow \mathcal{R}$};
			\node[main node] (B1) [left = 0cm of B] {};
			\node[main node] (G1) [fill=green!50,left = 0cm of G] {};
			\node[main node] (R1) [fill=red!50,left = 0cm of R] {};

			\node (text1) [above right = 0cm and 4cm of 7] {\enquote{Partitionierung}};
			\node (text2) [below = 0.1cm of text1] {\enquote{stabile Partitionierung}};
			\node (text3) [below = 0.1cm of text2] {\enquote{Zellen}};
		\end{tikzpicture}
	\end{frame}
	
	\section{Lokale Struktur}
	\begin{frame}{Lokale Struktur von CR-Graphen}
		\begin{Lemma}
			Die Zellen der stabilen Partition $\mathcal{P}_G$ eines CR-Graphen erfüllen folgende Eigenschaften:
			
			\begin{enumerate}[label=(\Alph*)]
				\item Für beliebige Zellen $X\in \mathcal{P}_G$ ist $G[X]$ ein leerer Graph, vollständiger Graph, Matching-Graph $mK_2$, das Komplement eines Matching Graphen oder der 5-Kreis.
			\end{enumerate}
		\end{Lemma}
	\end{frame}
	\begin{frame}{Am Beispiel \only<2>{- Leerer Graph}\only<3>{- Matching-Graph}}
		\centering
		\begin{tikzpicture}
			\def\scale{0.6}
			\begin{scope}
				\node[main node] (1) [fill=green!20]{A};
				\node[main node] (2) [fill=red!20,below left = \scale*0.57cm and \scale*1cm of 1]  {B};
				\node[main node] (6) [fill=red!20,below right = \scale*0.57cm and \scale*1cm of 1] {F};
				\node[main node] (3) [fill=red!20,below = \scale*1cm of 2] {C};
				\node[main node] (5) [fill=red!20,below = \scale*1cm of 6] {E};
				\node[main node] (4) [fill=green!20,below right = \scale*0.57cm and \scale*1cm of 3] {D};
				\node[main node] (7) [fill=blue!20,above = \scale*1cm of 1] {G};
				\node[main node] (8) [fill=blue!20,below = \scale*1cm of 4] {H};
				
				\path[draw=black!30]
				(1) edge node {} (2)
				(2) edge node {} (3)
				(3) edge node {} (4)
				(4) edge node {} (5)
				(5) edge node {} (6)
				(6) edge node {} (1)
				(1) edge node {} (7)
				(4) edge node {} (8);
				
				\node (B) [above right = -0.4cm and 2cm of 6] {$\rightarrow \mathcal{B}$};
				\node (G) [below = 0.2cm of B] {$\rightarrow \mathcal{G}$};
				\node (R) [below = 0.2cm of G] {$\rightarrow \mathcal{R}$};
				\node[main node] (B1) [fill=blue!20,left = 0cm of B] {};
				\node[main node] (G1) [fill=green!20,left = 0cm of G] {};
				\node[main node] (R1) [fill=red!20,left = 0cm of R] {};
			\end{scope}
			
			\pause
			
			\begin{scope}
				\node[main node] (1) [fill=green!80]{A};
				\node[main node] (2) [fill=black!20,below left = \scale*0.57cm and \scale*1cm of 1]  {B};
				\node[main node] (6) [fill=black!20,below right = \scale*0.57cm and \scale*1cm of 1] {F};
				\node[main node] (3) [fill=black!20,below = \scale*1cm of 2] {C};
				\node[main node] (5) [fill=black!20,below = \scale*1cm of 6] {E};
				\node[main node] (4) [fill=green!80,below right = \scale*0.57cm and \scale*1cm of 3] {D};
				\node[main node] (7) [fill=black!20,above = \scale*1cm of 1] {G};
				\node[main node] (8) [fill=black!20,below = \scale*1cm of 4] {H};
				
				\path[draw=black!30]
				(1) edge node {} (2)
				(2) edge node {} (3)
				(3) edge node {} (4)
				(4) edge node {} (5)
				(5) edge node {} (6)
				(6) edge node {} (1)
				(1) edge node {} (7)
				(4) edge node {} (8);
				
				\node (B) [above right = -0.4cm and 2cm of 6] {$\rightarrow \mathcal{B}$};
				\node (G) [below = 0.2cm of B] {$\rightarrow \mathcal{G}$};
				\node (R) [below = 0.2cm of G] {$\rightarrow \mathcal{R}$};
				\node[main node] (B1) [fill=black!20,left = 0cm of B] {};
				\node[main node] (G1) [fill=green!80,left = 0cm of G] {};
				\node[main node] (R1) [fill=black!20,left = 0cm of R] {};
			\end{scope}
			
			\pause
			
			\begin{scope}
				\node[main node] (1) [fill=black!20]{A};
				\node[main node] (2) [fill=red!80,below left = \scale*0.57cm and \scale*1cm of 1]  {B};
				\node[main node] (6) [fill=red!80,below right = \scale*0.57cm and \scale*1cm of 1] {F};
				\node[main node] (3) [fill=red!80,below = \scale*1cm of 2] {C};
				\node[main node] (5) [fill=red!80,below = \scale*1cm of 6] {E};
				\node[main node] (4) [fill=black!20,below right = \scale*0.57cm and \scale*1cm of 3] {D};
				\node[main node] (7) [fill=black!20,above = \scale*1cm of 1] {G};
				\node[main node] (8) [fill=black!20,below = \scale*1cm of 4] {H};
				
				\path[draw=black!30]
				(1) edge node {} (2)
				(3) edge node {} (4)
				(4) edge node {} (5)
				(6) edge node {} (1)
				(1) edge node {} (7)
				(4) edge node {} (8);
				
				\path[draw,line width=1mm]
				(2) edge node {} (3)
				(5) edge node {} (6);
				
				\node (B) [above right = -0.4cm and 2cm of 6] {$\rightarrow \mathcal{B}$};
				\node (G) [below = 0.2cm of B] {$\rightarrow \mathcal{G}$};
				\node (R) [below = 0.2cm of G] {$\rightarrow \mathcal{R}$};
				\node[main node] (B1) [fill=black!20,left = 0cm of B] {};
				\node[main node] (G1) [fill=black!20,left = 0cm of G] {};
				\node[main node] (R1) [fill=red!80,left = 0cm of R] {};
			\end{scope}
		\end{tikzpicture}
	\end{frame}
	\begin{frame}{Lokale Struktur von CR-Graphen}
		\begin{Lemma}
			Die Zellen der stabilen Partition $\mathcal{P}_G$ eines CR-Graphen erfüllen folgende Eigenschaften:
			
			\begin{enumerate}[label=(\Alph*)]
				\item Für beliebige Zellen $X\in \mathcal{P}_G$ ist $G[X]$ ein leerer Graph, vollständiger Graph, Matching-Graph $mK_2$, das Komplement eines Matching Graphen oder der 5-Kreis.
				\pause
				\item Für beliebige Zellen $X,Y\in \mathcal{P}_G$ ist $G[X,Y]$ ein leerer Graph, vollständiger bipartiter Graph, eine disjunkte Vereinigung von Sternen $sK_{1,t}$, bei der $X$ die Menge der $s$ inneren Knoten und $Y$ die Menge der $st$ Blätter ist, oder das bipartite Komplement des zuletzt genannten Graphen.
			\end{enumerate}
		\end{Lemma}
	\end{frame}
	\begin{frame}{Am Beispiel \only<2>{- Leerer Graph}\only<3>{- Disjunkte Vereinigung von Sternen $sK_{1,t}$}\only<4>{- Matching Graph*}}
		\centering
		\begin{tikzpicture}
			\def\scale{0.6}
			\only<1>{\begin{scope}
				\node[main node] (1) [fill=green!20]{A};
				\node[main node] (2) [fill=red!20,below left = \scale*0.57cm and \scale*1cm of 1]  {B};
				\node[main node] (6) [fill=red!20,below right = \scale*0.57cm and \scale*1cm of 1] {F};
				\node[main node] (3) [fill=red!20,below = \scale*1cm of 2] {C};
				\node[main node] (5) [fill=red!20,below = \scale*1cm of 6] {E};
				\node[main node] (4) [fill=green!20,below right = \scale*0.57cm and \scale*1cm of 3] {D};
				\node[main node] (7) [fill=blue!20,above = \scale*1cm of 1] {G};
				\node[main node] (8) [fill=blue!20,below = \scale*1cm of 4] {H};
				
				\path[draw=black!30]
				(1) edge node {} (2)
				(2) edge node {} (3)
				(3) edge node {} (4)
				(4) edge node {} (5)
				(5) edge node {} (6)
				(6) edge node {} (1)
				(1) edge node {} (7)
				(4) edge node {} (8);
				
				\node (B) [above right = -0.4cm and 2cm of 6] {$\rightarrow \mathcal{B}$};
				\node (G) [below = 0.2cm of B] {$\rightarrow \mathcal{G}$};
				\node (R) [below = 0.2cm of G] {$\rightarrow \mathcal{R}$};
				\node[main node] (B1) [fill=blue!20,left = 0cm of B] {};
				\node[main node] (G1) [fill=green!20,left = 0cm of G] {};
				\node[main node] (R1) [fill=red!20,left = 0cm of R] {};
			\end{scope}}
			
			\only<2>{\begin{scope}
				\node[main node] (1) [fill=black!20]{A};
				\node[main node] (2) [fill=red!50,below left = \scale*0.57cm and \scale*1cm of 1]  {B};
				\node[main node] (6) [fill=red!50,below right = \scale*0.57cm and \scale*1cm of 1] {F};
				\node[main node] (3) [fill=red!50,below = \scale*1cm of 2] {C};
				\node[main node] (5) [fill=red!50,below = \scale*1cm of 6] {E};
				\node[main node] (4) [fill=black!20,below right = \scale*0.57cm and \scale*1cm of 3] {D};
				\node[main node] (7) [fill=blue!50,above = \scale*1cm of 1] {G};
				\node[main node] (8) [fill=blue!50,below = \scale*1cm of 4] {H};
				
				\path[draw=black!30]
				(1) edge node {} (2)
				(2) edge node {} (3)
				(3) edge node {} (4)
				(4) edge node {} (5)
				(5) edge node {} (6)
				(6) edge node {} (1)
				(1) edge node {} (7)
				(4) edge node {} (8);
				
				\node (B) [above right = -0.4cm and 2cm of 6] {$\rightarrow \mathcal{B}$};
				\node (G) [below = 0.2cm of B] {$\rightarrow \mathcal{G}$};
				\node (R) [below = 0.2cm of G] {$\rightarrow \mathcal{R}$};
				\node[main node] (B1) [fill=blue!50,left = 0cm of B] {};
				\node[main node] (G1) [fill=black!20,left = 0cm of G] {};
				\node[main node] (R1) [fill=red!50,left = 0cm of R] {};
			\end{scope}}
			
			\only<3>{\begin{scope}
				\node[main node] (1) [fill=green!50]{A};
				\node[main node] (2) [fill=red!50,below left = \scale*0.57cm and \scale*1cm of 1]  {B};
				\node[main node] (6) [fill=red!50,below right = \scale*0.57cm and \scale*1cm of 1] {F};
				\node[main node] (3) [fill=red!50,below = \scale*1cm of 2] {C};
				\node[main node] (5) [fill=red!50,below = \scale*1cm of 6] {E};
				\node[main node] (4) [fill=green!50,below right = \scale*0.57cm and \scale*1cm of 3] {D};
				\node[main node] (7) [fill=black!20,above = \scale*1cm of 1] {G};
				\node[main node] (8) [fill=black!20,below = \scale*1cm of 4] {H};
				
				\path[draw=black!30]
				(2) edge node {} (3)
				(5) edge node {} (6)
				(1) edge node {} (7)
				(4) edge node {} (8);
				
				\path[draw,line width=1mm]
				(1) edge node {} (2)
				(6) edge node {} (1)
				(3) edge node {} (4)
				(4) edge node {} (5);
				
				\node (B) [above right = -0.4cm and 2cm of 6] {$\rightarrow \mathcal{B}$};
				\node (G) [below = 0.2cm of B] {$\rightarrow \mathcal{G}$};
				\node (R) [below = 0.2cm of G] {$\rightarrow \mathcal{R}$};
				\node[main node] (B1) [fill=black!20,left = 0cm of B] {};
				\node[main node] (G1) [fill=black!20,left = 0cm of G] {};
				\node[main node] (R1) [fill=black!20,left = 0cm of R] {};
			\end{scope}}
			
			\only<4>{\begin{scope}
				\node[main node] (1) [fill=green!50]{A};
				\node[main node] (2) [fill=black!20,below left = \scale*0.57cm and \scale*1cm of 1]  {B};
				\node[main node] (6) [fill=black!20,below right = \scale*0.57cm and \scale*1cm of 1] {F};
				\node[main node] (3) [fill=black!20,below = \scale*1cm of 2] {C};
				\node[main node] (5) [fill=black!20,below = \scale*1cm of 6] {E};
				\node[main node] (4) [fill=green!50,below right = \scale*0.57cm and \scale*1cm of 3] {D};
				\node[main node] (7) [fill=blue!50,above = \scale*1cm of 1] {G};
				\node[main node] (8) [fill=blue!50,below = \scale*1cm of 4] {H};
				
				\path[draw=black!30]
				(1) edge node {} (2)
				(2) edge node {} (3)
				(3) edge node {} (4)
				(4) edge node {} (5)
				(5) edge node {} (6)
				(6) edge node {} (1);
				
				\path[draw,line width=1mm]
				(1) edge node {} (7)
				(4) edge node {} (8);
				
				\node (B) [above right = -0.4cm and 2cm of 6] {$\rightarrow \mathcal{B}$};
				\node (G) [below = 0.2cm of B] {$\rightarrow \mathcal{G}$};
				\node (R) [below = 0.2cm of G] {$\rightarrow \mathcal{R}$};
				\node[main node] (B1) [fill=blue!50,left = 0cm of B] {};
				\node[main node] (G1) [fill=green!50,left = 0cm of G] {};
				\node[main node] (R1) [fill=black!20,left = 0cm of R] {};
			\end{scope}}
		\end{tikzpicture}
	\end{frame}
	\begin{frame}{Lokale Struktur von CR-Graphen}
		\begin{Lemma}
			Die Zellen der stabilen Partition $\mathcal{P}_G$ eines CR-Graphen erfüllen folgende Eigenschaften:
			
			\begin{enumerate}[label=(\Alph*)]
				\item Für beliebige Zellen $X\in \mathcal{P}_G$ ist $G[X]$ ein leerer Graph, vollständiger Graph, Matching-Graph $mK_2$, das Komplement eines Matching Graphen oder der 5-Kreis.
				\item Für beliebige Zellen $X,Y\in \mathcal{P}_G$ ist $G[X,Y]$ ein leerer Graph, vollständiger bipartiter Graph, eine disjunkte Vereinigung von Sternen $sK_{1,t}$, bei der $X$ die Menge der $s$ inneren Knoten und $Y$ die Menge der $st$ Blätter ist, oder das bipartite Komplement des zuletzt genannten Graphen.
			\end{enumerate}
		\end{Lemma}
	\end{frame}

	\section{Globale Struktur}
	\begin{frame}{Zellgraph}
		\begin{Definition}
			Der Zellgraph $C(G)$ eines Graphen $G$ wird aus dessen stabilen Partition $\mathcal{P}_G$ gebildet.
			Es handelt sich dabei um einen vollständigen Graphen, bei dem die Knoten die Zellen von $\mathcal{P}_G$ darstellen.
		\end{Definition}
	\end{frame}
	\begin{frame}{Zellgraph - Beispiel}
		\centering
		\begin{tikzpicture}
		\def\scale{0.6}
		\begin{scope}
			\node[main node] (1) [fill=green!50]{A};
			\node[main node] (2) [fill=red!50,below left = \scale*0.57cm and \scale*1cm of 1]  {B};
			\node[main node] (6) [fill=red!50,below right = \scale*0.57cm and \scale*1cm of 1] {F};
			\node[main node] (3) [fill=red!50,below = \scale*1cm of 2] {C};
			\node[main node] (5) [fill=red!50,below = \scale*1cm of 6] {E};
			\node[main node] (4) [fill=green!50,below right = \scale*0.57cm and \scale*1cm of 3] {D};
			\node[main node] (7) [above = \scale*1cm of 1] {G};
			\node[main node] (8) [below = \scale*1cm of 4] {H};
			
			\path[draw,thick]
			(1) edge node {} (2)
			(2) edge node {} (3)
			(3) edge node {} (4)
			(4) edge node {} (5)
			(5) edge node {} (6)
			(6) edge node {} (1)
			(1) edge node {} (7)
			(4) edge node {} (8);
		\end{scope}
		
		\begin{scope}[yshift=-0.8cm,xshift=5cm]
			\node[main node] (1) [fill=blue!50] {};
			\node[main node] (2) [fill=green!50,below left = 0.5cm and 0.85cm of 1] {};
			\node[main node] (3) [fill=red!50,below = 1cm of 1] {};
			
			\path[draw,thick] (1) edge node {} (2);
			\path[draw,thick] (2) edge node {} (3);
			\path[draw,thick] (1) edge node {} (3);
		\end{scope}
%		\begin{scope}[xshift=6cm]
%			\node[main node] (1) [fill=white] {1};
%			\node[main node] (2) [fill=white,dashed,below = 0.15cm of 1] {2};
%			
%			\node (t1) [right = 0.2cm of 1] {homogene Zelle};
%			\node (t2) [right = 0.2cm of 2] {heterogene Zelle};
%			
%			\node[bezier] (31) [below left = 0.55cm and 0.15cm of 2] {};
%			\node[bezier] (32) [right = 0.85cm of 31] {};
%			\path[draw,thick] (31) edge node {} (32);
%			
%			\node[bezier] (41) [below = 0.75cm of 31] {};
%			\node[bezier] (42) [right = 0.85cm of 41] {};
%			\path[draw,thick,dashed] (41) edge node {} (42);
%			
%			\node (t3) [right = 0.12cm of 32] {isotrope Kante};
%			\node (t4) [right = 0.12cm of 42] {anisotrope Kante};
%			
%			\node[main node] (5) [fill = white, below = 1.6cm of 2] {$x$};
%			\node (t5) [right = 0.2cm of 5] {Kardinalität $x$};
%		\end{scope}
		\end{tikzpicture}
	\end{frame}
	\begin{frame}{Eigenschaften von Zellgraphen}
		\begin{Definition}
			Eine Zelle $X\in C(G)$ wird \alert{\textbf{homogen}} genannt, wenn der Graph $G[X]$ vollständig oder leer ist.
			Anderenfalls wird diese \alert{\textbf{heterogen}} genannt.
		\end{Definition}
		\pause
		\begin{Definition}
			Eine Kante $\{X,Y\}$ mit $X,Y\in C(G)$ wird \alert{\textbf{isotrop}} genannt, wenn der bipartite Graph $G[X,Y]$ vollständig oder leer ist.
			Anderenfalls wird diese \alert{\textbf{anisotrop}} genannt.
		\end{Definition}
		\pause
		\begin{Definition}
			Ein Pfad $X_1X_2...X_l$ in $C(G)$, bei der jede Kante anisotrop ist, wird \alert{\textbf{anisotroper Pfad}} genannt.
			Wenn dieser Pfad einen Kreis schließt wird er als \alert{\textbf{anisotroper Zyklus}} bezeichnet.
			Gilt für einen anisotropen Pfad $|X_1|=|X_2|=...=|X_l|$, dann wird er \alert{\textbf{gleichmäßig}} genannt.
		\end{Definition}
	\end{frame}
	\begin{frame}{Eigenschaften von Zellgraphen - Beispiel}
		\centering
		\begin{tikzpicture}
			\def\scale{0.6}
			\begin{scope}
				\node[main node] (1) [fill=green!50]{A};
				\node[main node] (2) [fill=red!50,below left = \scale*0.57cm and \scale*1cm of 1]  {B};
				\node[main node] (6) [fill=red!50,below right = \scale*0.57cm and \scale*1cm of 1] {F};
				\node[main node] (3) [fill=red!50,below = \scale*1cm of 2] {C};
				\node[main node] (5) [fill=red!50,below = \scale*1cm of 6] {E};
				\node[main node] (4) [fill=green!50,below right = \scale*0.57cm and \scale*1cm of 3] {D};
				\node[main node] (7) [above = \scale*1cm of 1] {G};
				\node[main node] (8) [below = \scale*1cm of 4] {H};
				
				\path[draw,thick]
				(1) edge node {} (2)
				(2) edge node {} (3)
				(3) edge node {} (4)
				(4) edge node {} (5)
				(5) edge node {} (6)
				(6) edge node {} (1)
				(1) edge node {} (7)
				(4) edge node {} (8);
			\end{scope}
			
			\begin{scope}[yshift=-0.8cm,xshift=4cm]
				\node[main node] (1) [fill=blue!50] {$2$};
				\node[main node] (2) [fill=green!50,below left = 0.5cm and 0.85cm of 1] {$2$};
				\node[main node] (3) [dashed,fill=red!50,below = 1cm of 1] {$4$};
				
				\path[draw,thick,dashed] (1) edge node {} (2);
				\path[draw,thick,dashed] (2) edge node {} (3);
				\path[draw,thick] (1) edge node {} (3);
			\end{scope}
			\begin{scope}[xshift=6cm]
				\node[main node] (1) [fill=white] {};
				\node[main node] (2) [fill=white,dashed,below = 0.15cm of 1] {};
				
				\node (t1) [right = 0.2cm of 1] {homogene Zelle};
				\node (t2) [right = 0.2cm of 2] {heterogene Zelle};
				
				\node[bezier] (31) [below left = 0.55cm and 0.15cm of 2] {};
				\node[bezier] (32) [right = 0.85cm of 31] {};
				\path[draw,thick] (31) edge node {} (32);
				
				\node[bezier] (41) [below = 0.75cm of 31] {};
				\node[bezier] (42) [right = 0.85cm of 41] {};
				\path[draw,thick,dashed] (41) edge node {} (42);
				
				\node (t3) [right = 0.12cm of 32] {isotrope Kante};
				\node (t4) [right = 0.12cm of 42] {anisotrope Kante};
				
				\node[main node] (5) [fill = white, below = 1.6cm of 2] {$x$};
				\node (t5) [right = 0.2cm of 5] {Kardinalität $x$};
			\end{scope}
		\end{tikzpicture}
	\end{frame}
	\begin{frame}{Allgemeine Globale Eigenschaften von CR-Graphen}
		\begin{Lemma}
			Der Zellgraph $C(G)$ eines CR-Graphen $G$ erfüllt folgende Eigenschaften:
			
			\begin{enumerate}[label=(\Alph*)]
				\setcounter{enumi}{2}
				\item $C(G)$ enthält keinen gleichmäßigen, anisotropen Pfad, der zwei heterogene Zellen verbindet.
%				\item $C(G)$ enthält keinen gleichmäßigen, anisotropen Zyklus.
%				\item $C(G)$ enthält weder einen anisotropen Pfad $XY_1Y_2...Y_lZ$, sodass $|X|<|Y_1|=|Y_2|=...=|Y_l|>|Z|$, noch einen anisotropen Zyklus $XY_1Y_2...Y_l$, sodass $|X|<|Y_1|=|Y_2|=...=|Y_l|$ und die Zelle $Y_l$ heterogen ist.
%				\item $C(G)$ enthält keinen anisotropen Pfad $XY_1Y_2...Y_l$, sodass $|X|<|Y_1|=|Y_2|=...=|Y_l|$ und die Zelle $Y_l$ heterogen ist.
			\end{enumerate}
		\end{Lemma}
	\end{frame}
	\begin{frame}{Allgemeine Globale Eigenschaften von CR-Graphen}
		\centering
		\textit{$C(G)$ enthält keinen gleichmäßigen, anisotropen Pfad, der zwei heterogene Zellen verbindet.}
		
		\begin{tikzpicture}
			\begin{scope}
				\node[main node] (1) [fill=blue!50] {$2$};
				\node[main node] (2) [fill=green!50,below left = 0.5cm and 0.85cm of 1] {$2$};
				\node[main node] (3) [dashed,fill=red!50,below = 1cm of 1] {$4$};
				
				\path[draw,thick,dashed] (1) edge node {} (2);
				\path[draw,thick,dashed] (2) edge node {} (3);
				\path[draw,thick] (1) edge node {} (3);
			\end{scope}
			\begin{scope}[xshift=2cm]
				\node[main node] (1) [fill=white] {};
				\node[main node] (2) [fill=white,dashed,below = 0.15cm of 1] {};
				
				\node (t1) [right = 0.2cm of 1] {homogene Zelle};
				\node (t2) [right = 0.2cm of 2] {heterogene Zelle};
				
				\node[bezier] (31) [below left = 0.55cm and 0.15cm of 2] {};
				\node[bezier] (32) [right = 0.85cm of 31] {};
				\path[draw,thick] (31) edge node {} (32);
				
				\node[bezier] (41) [below = 0.75cm of 31] {};
				\node[bezier] (42) [right = 0.85cm of 41] {};
				\path[draw,thick,dashed] (41) edge node {} (42);
				
				\node (t3) [right = 0.12cm of 32] {isotrope Kante};
				\node (t4) [right = 0.12cm of 42] {anisotrope Kante};
				
				\node[main node] (5) [fill = white, below = 1.6cm of 2] {$x$};
				\node (t5) [right = 0.2cm of 5] {Kardinalität $x$};
			\end{scope}
		\end{tikzpicture}
	\end{frame}
	\begin{frame}{Allgemeine Globale Eigenschaften von CR-Graphen}
		\begin{Lemma}
			Der Zellgraph $C(G)$ eines CR-Graphen $G$ erfüllt folgende Eigenschaften:
			
			\begin{enumerate}[label=(\Alph*)]
				\setcounter{enumi}{2}
				\item $C(G)$ enthält keinen gleichmäßigen, anisotropen Pfad, der zwei heterogene Zellen verbindet.
				\item $C(G)$ enthält keinen gleichmäßigen, anisotropen Zyklus.
%				\item $C(G)$ enthält weder einen anisotropen Pfad $XY_1Y_2...Y_lZ$, sodass $|X|<|Y_1|=|Y_2|=...=|Y_l|>|Z|$, noch einen anisotropen Zyklus $XY_1Y_2...Y_l$, sodass $|X|<|Y_1|=|Y_2|=...=|Y_l|$ und die Zelle $Y_l$ heterogen ist.
%				\item $C(G)$ enthält keinen anisotropen Pfad $XY_1Y_2...Y_l$, sodass $|X|<|Y_1|=|Y_2|=...=|Y_l|$ und die Zelle $Y_l$ heterogen ist.
			\end{enumerate}
		\end{Lemma}
	\end{frame}
	\begin{frame}{Allgemeine Globale Eigenschaften von CR-Graphen}
		\centering
		\textit{$C(G)$ enthält keinen gleichmäßigen, anisotropen Zyklus.}
		
		\begin{tikzpicture}
			\begin{scope}
				\node[main node] (1) [fill=blue!50] {$2$};
				\node[main node] (2) [fill=green!50,below left = 0.5cm and 0.85cm of 1] {$2$};
				\node[main node] (3) [dashed,fill=red!50,below = 1cm of 1] {$4$};
				
				\path[draw,thick,dashed] (1) edge node {} (2);
				\path[draw,thick,dashed] (2) edge node {} (3);
				\path[draw,thick] (1) edge node {} (3);
			\end{scope}
			\begin{scope}[xshift=2cm]
				\node[main node] (1) [fill=white] {};
				\node[main node] (2) [fill=white,dashed,below = 0.15cm of 1] {};
				
				\node (t1) [right = 0.2cm of 1] {homogene Zelle};
				\node (t2) [right = 0.2cm of 2] {heterogene Zelle};
				
				\node[bezier] (31) [below left = 0.55cm and 0.15cm of 2] {};
				\node[bezier] (32) [right = 0.85cm of 31] {};
				\path[draw,thick] (31) edge node {} (32);
				
				\node[bezier] (41) [below = 0.75cm of 31] {};
				\node[bezier] (42) [right = 0.85cm of 41] {};
				\path[draw,thick,dashed] (41) edge node {} (42);
				
				\node (t3) [right = 0.12cm of 32] {isotrope Kante};
				\node (t4) [right = 0.12cm of 42] {anisotrope Kante};
				
				\node[main node] (5) [fill = white, below = 1.6cm of 2] {$x$};
				\node (t5) [right = 0.2cm of 5] {Kardinalität $x$};
			\end{scope}
		\end{tikzpicture}
	\end{frame}
	\begin{frame}{Allgemeine Globale Eigenschaften von CR-Graphen}
		\begin{Lemma}
			Der Zellgraph $C(G)$ eines CR-Graphen $G$ erfüllt folgende Eigenschaften:
			
			\begin{enumerate}[label=(\Alph*)]
				\setcounter{enumi}{2}
				\item $C(G)$ enthält keinen gleichmäßigen, anisotropen Pfad, der zwei heterogene Zellen verbindet.
				\item $C(G)$ enthält keinen gleichmäßigen, anisotropen Zyklus.
				\item $C(G)$ enthält weder einen anisotropen Pfad $XY_1Y_2...Y_lZ$, sodass $|X|<|Y_1|=|Y_2|=...=|Y_l|>|Z|$, noch einen anisotropen Zyklus $XY_1Y_2...Y_l$, sodass $|X|<|Y_1|=|Y_2|=...=|Y_l|$ und die Zelle $Y_l$ heterogen ist.
%				\item $C(G)$ enthält keinen anisotropen Pfad $XY_1Y_2...Y_l$, sodass $|X|<|Y_1|=|Y_2|=...=|Y_l|$ und die Zelle $Y_l$ heterogen ist.
			\end{enumerate}
		\end{Lemma}
	\end{frame}
	\begin{frame}{Allgemeine Globale Eigenschaften von CR-Graphen}
		\centering
		\textit{$C(G)$ enthält weder einen anisotropen Pfad $XY_1Y_2...Y_lZ$, sodass $|X|<|Y_1|=|Y_2|=...=|Y_l|>|Z|$, noch einen anisotropen Zyklus $XY_1Y_2...Y_l$, sodass $|X|<|Y_1|=|Y_2|=...=|Y_l|$ und die Zelle $Y_l$ heterogen ist.}
		
		\begin{tikzpicture}
			\begin{scope}
				\node[main node] (1) [fill=blue!50] {$2$};
				\node[main node] (2) [fill=green!50,below left = 0.5cm and 0.85cm of 1] {$2$};
				\node[main node] (3) [dashed,fill=red!50,below = 1cm of 1] {$4$};
				
				\path[draw,thick,dashed] (1) edge node {} (2);
				\path[draw,thick,dashed] (2) edge node {} (3);
				\path[draw,thick] (1) edge node {} (3);
			\end{scope}
			\begin{scope}[xshift=2cm]
				\node[main node] (1) [fill=white] {};
				\node[main node] (2) [fill=white,dashed,below = 0.15cm of 1] {};
				
				\node (t1) [right = 0.2cm of 1] {homogene Zelle};
				\node (t2) [right = 0.2cm of 2] {heterogene Zelle};
				
				\node[bezier] (31) [below left = 0.55cm and 0.15cm of 2] {};
				\node[bezier] (32) [right = 0.85cm of 31] {};
				\path[draw,thick] (31) edge node {} (32);
				
				\node[bezier] (41) [below = 0.75cm of 31] {};
				\node[bezier] (42) [right = 0.85cm of 41] {};
				\path[draw,thick,dashed] (41) edge node {} (42);
				
				\node (t3) [right = 0.12cm of 32] {isotrope Kante};
				\node (t4) [right = 0.12cm of 42] {anisotrope Kante};
				
				\node[main node] (5) [fill = white, below = 1.6cm of 2] {$x$};
				\node (t5) [right = 0.2cm of 5] {Kardinalität $x$};
			\end{scope}
		\end{tikzpicture}
	\end{frame}
	\begin{frame}{Allgemeine Globale Eigenschaften von CR-Graphen}
		\begin{Lemma}
			Der Zellgraph $C(G)$ eines CR-Graphen $G$ erfüllt folgende Eigenschaften:
			
			\begin{enumerate}[label=(\Alph*)]
				\setcounter{enumi}{2}
				\item $C(G)$ enthält keinen gleichmäßigen, anisotropen Pfad, der zwei heterogene Zellen verbindet.
				\item $C(G)$ enthält keinen gleichmäßigen, anisotropen Zyklus.
				\item $C(G)$ enthält weder einen anisotropen Pfad $XY_1Y_2...Y_lZ$, sodass $|X|<|Y_1|=|Y_2|=...=|Y_l|>|Z|$, noch einen anisotropen Zyklus $XY_1Y_2...Y_l$, sodass $|X|<|Y_1|=|Y_2|=...=|Y_l|$ und die Zelle $Y_l$ heterogen ist.
				\item $C(G)$ enthält keinen anisotropen Pfad $XY_1Y_2...Y_l$, sodass $|X|<|Y_1|=|Y_2|=...=|Y_l|$ und die Zelle $Y_l$ heterogen ist.
			\end{enumerate}
		\end{Lemma}
	\end{frame}
	\begin{frame}{Allgemeine Globale Eigenschaften von CR-Graphen}
		\centering
		\textit{$C(G)$ enthält keinen anisotropen Pfad $XY_1Y_2...Y_l$, sodass $|X|<|Y_1|=|Y_2|=...=|Y_l|$ und die Zelle $Y_l$ heterogen ist.}
		
		\begin{tikzpicture}
			\only<1>{\begin{scope}
				\node[main node] (1) [fill=blue!50] {$2$};
				\node[main node] (2) [fill=green!50,below left = 0.5cm and 0.85cm of 1] {$2$};
				\node[main node] (3) [dashed,fill=red!50,below = 1cm of 1] {$4$};
				
				\path[draw,thick,dashed] (1) edge node {} (2);
				\path[draw,thick,dashed] (2) edge node {} (3);
				\path[draw,thick] (1) edge node {} (3);
			\end{scope}
			\begin{scope}[xshift=2cm]
				\node[main node] (1) [fill=white] {};
				\node[main node] (2) [fill=white,dashed,below = 0.15cm of 1] {};
				
				\node (t1) [right = 0.2cm of 1] {homogene Zelle};
				\node (t2) [right = 0.2cm of 2] {heterogene Zelle};
				
				\node[bezier] (31) [below left = 0.55cm and 0.15cm of 2] {};
				\node[bezier] (32) [right = 0.85cm of 31] {};
				\path[draw,thick] (31) edge node {} (32);
				
				\node[bezier] (41) [below = 0.75cm of 31] {};
				\node[bezier] (42) [right = 0.85cm of 41] {};
				\path[draw,thick,dashed] (41) edge node {} (42);
				
				\node (t3) [right = 0.12cm of 32] {isotrope Kante};
				\node (t4) [right = 0.12cm of 42] {anisotrope Kante};
				
				\node[main node] (5) [fill = white, below = 1.6cm of 2] {$x$};
				\node (t5) [right = 0.2cm of 5] {Kardinalität $x$};
			\end{scope}}

			\only<2>{\begin{scope}
				\node[main node] (1) [fill=black!20] {$2$};
				\node[main node] (2) [fill=green!50,below left = 0.5cm and 0.85cm of 1] {$2$};
				\node[main node] (3) [dashed,fill=red!50,below = 1cm of 1] {$4$};
				
				\path[draw,thick,dashed,draw=black!30] (1) edge node {} (2);
				\path[draw,thick,dashed] (2) edge node {} (3);
				\path[draw,thick,draw=black!30] (1) edge node {} (3);
			\end{scope}
			\begin{scope}[xshift=2cm]
				\node[main node] (1) [fill=white] {};
				\node[main node] (2) [fill=white,dashed,below = 0.15cm of 1] {};
				
				\node (t1) [right = 0.2cm of 1] {homogene Zelle};
				\node (t2) [right = 0.2cm of 2] {heterogene Zelle};
				
				\node[bezier] (31) [below left = 0.55cm and 0.15cm of 2] {};
				\node[bezier] (32) [right = 0.85cm of 31] {};
				\path[draw,thick] (31) edge node {} (32);
				
				\node[bezier] (41) [below = 0.75cm of 31] {};
				\node[bezier] (42) [right = 0.85cm of 41] {};
				\path[draw,thick,dashed] (41) edge node {} (42);
				
				\node (t3) [right = 0.12cm of 32] {isotrope Kante};
				\node (t4) [right = 0.12cm of 42] {anisotrope Kante};
				
				\node[main node] (5) [fill = white, below = 1.6cm of 2] {$x$};
				\node (t5) [right = 0.2cm of 5] {Kardinalität $x$};
			\end{scope}
			\begin{scope}[yshift=-3cm,xshift=-1cm]
				\node {\Huge\Lightning};
			\end{scope}}
		\end{tikzpicture}
	\end{frame}
	\begin{frame}{Definition}
		\centering
		\begin{Definition}
			In einem Zellgraphen $C(G)$ bezeichnet eine \alert{\textbf{anisotrope Komponente}} einen Subgraphen, dessen Kanten alle anisotrop sind.
		\end{Definition}
		\pause
		\begin{tikzpicture}
			\only<2>{\begin{scope}
				\node[main node] (1) [fill=blue!50] {$2$};
				\node[main node] (2) [fill=black!20,below left = 0.5cm and 0.85cm of 1] {$2$};
				\node[main node] (3) [dashed,fill=black!20,below = 1cm of 1] {$4$};
				
				\path[draw,thick,dashed,draw=black!30] (1) edge node {} (2);
				\path[draw,thick,dashed,draw=black!30] (2) edge node {} (3);
				\path[draw,thick,draw=black!30] (1) edge node {} (3);
			\end{scope}}
			\only<3>{\begin{scope}
				\node[main node] (1) [fill=black!20] {$2$};
				\node[main node] (2) [fill=green!50,below left = 0.5cm and 0.85cm of 1] {$2$};
				\node[main node] (3) [dashed,fill=black!20,below = 1cm of 1] {$4$};
				
				\path[draw,thick,dashed,draw=black!30] (1) edge node {} (2);
				\path[draw,thick,dashed,draw=black!30] (2) edge node {} (3);
				\path[draw,thick,draw=black!30] (1) edge node {} (3);
			\end{scope}}
			\only<4>{\begin{scope}
				\node[main node] (1) [fill=black!20] {$2$};
				\node[main node] (2) [fill=black!20,below left = 0.5cm and 0.85cm of 1] {$2$};
				\node[main node] (3) [dashed,fill=red!50,below = 1cm of 1] {$4$};
				
				\path[draw,thick,dashed,draw=black!30] (1) edge node {} (2);
				\path[draw,thick,dashed,draw=black!30] (2) edge node {} (3);
				\path[draw,thick,draw=black!30] (1) edge node {} (3);
			\end{scope}}
			\only<5>{\begin{scope}
				\node[main node] (1) [fill=blue!50] {$2$};
				\node[main node] (2) [fill=green!50,below left = 0.5cm and 0.85cm of 1] {$2$};
				\node[main node] (3) [dashed,fill=black!20,below = 1cm of 1] {$4$};
				
				\path[draw,thick,dashed] (1) edge node {} (2);
				\path[draw,thick,dashed,draw=black!30] (2) edge node {} (3);
				\path[draw,thick,draw=black!30] (1) edge node {} (3);
			\end{scope}}
			\only<6>{\begin{scope}
				\node[main node] (1) [fill=black!20] {$2$};
				\node[main node] (2) [fill=green!50,below left = 0.5cm and 0.85cm of 1] {$2$};
				\node[main node] (3) [dashed,fill=red!50,below = 1cm of 1] {$4$};
				
				\path[draw,thick,dashed,draw=black!30] (1) edge node {} (2);
				\path[draw,thick,dashed] (2) edge node {} (3);
				\path[draw,thick,draw=black!30] (1) edge node {} (3);
			\end{scope}}
			\only<7>{\begin{scope}
				\node[main node] (1) [fill=blue!50] {$2$};
				\node[main node] (2) [fill=green!50,below left = 0.5cm and 0.85cm of 1] {$2$};
				\node[main node] (3) [dashed,fill=red!50,below = 1cm of 1] {$4$};
				
				\path[draw,thick,dashed] (1) edge node {} (2);
				\path[draw,thick,dashed] (2) edge node {} (3);
				\path[draw,thick,draw=black!30] (1) edge node {} (3);
			\end{scope}}
			
			
			
			\begin{scope}[xshift=2cm]
				\node[main node] (1) [fill=white] {};
				\node[main node] (2) [fill=white,dashed,below = 0.15cm of 1] {};
				
				\node (t1) [right = 0.2cm of 1] {homogene Zelle};
				\node (t2) [right = 0.2cm of 2] {heterogene Zelle};
				
				\node[bezier] (31) [below left = 0.55cm and 0.15cm of 2] {};
				\node[bezier] (32) [right = 0.85cm of 31] {};
				\path[draw,thick] (31) edge node {} (32);
				
				\node[bezier] (41) [below = 0.75cm of 31] {};
				\node[bezier] (42) [right = 0.85cm of 41] {};
				\path[draw,thick,dashed] (41) edge node {} (42);
				
				\node (t3) [right = 0.12cm of 32] {isotrope Kante};
				\node (t4) [right = 0.12cm of 42] {anisotrope Kante};
				
				\node[main node] (5) [fill = white, below = 1.6cm of 2] {$x$};
				\node (t5) [right = 0.2cm of 5] {Kardinalität $x$};
			\end{scope}
		\end{tikzpicture}
	\end{frame}
	\begin{frame}{Baumartige Struktur von CR-Graphen}
		\begin{Lemma}
			Angenommen ein CR-Graph $G$ erfüllt die Bedingungen \textbf{A-F}.
			Für jede anisotrope Komponente $A$ von $C(G)$ gelten folgende Eigenschaften:
			
			\begin{enumerate}[label=(\Alph*)]
				\setcounter{enumi}{6}
				\item $A$ ist ein Baum, der folgende Monotonieeigenschaft erfüllt: Sei $R$ eine Zelle aus $A$ mit minimaler Kardinalität, so ist $A_R$ der gerichtete Baum mit Wurzel $R$; Für jede gerichtete Kante $(X,Y)$ aus $A_R$ gilt dann $|X|\leq |Y|$.
%				\item $A$ enthält maximal eine heterogene Zelle; Wenn eine solche Zelle existiert, hat diese minimale Kardinalität in $A$.
			\end{enumerate}
			\label{lemma:global2}
		\end{Lemma}
	\end{frame}
	\begin{frame}{Baumartige Struktur von CR-Graphen}
		\centering
		\textit{$A$ ist ein Baum, der folgende Monotonieeigenschaft erfüllt: Sei $R$ eine Zelle aus $A$ mit minimaler Kardinalität, so ist $A_R$ der gerichtete Baum mit Wurzel $R$; Für jede gerichtete Kante $(X,Y)$ aus $A_R$ gilt dann $|X|\leq |Y|$.}
		
		\begin{tikzpicture}
			\begin{scope}
				\node[main node] (1) [fill=blue!50] {$2$};
				\node[main node] (2) [fill=green!50,below left = 0.5cm and 0.85cm of 1] {$2$};
				\node[main node] (3) [dashed,fill=red!50,below = 1cm of 1] {$4$};
				
				\path[draw,thick,dashed] (1) edge node {} (2);
				\path[draw,thick,dashed] (2) edge node {} (3);
				\path[draw,thick] (1) edge node {} (3);
			\end{scope}
			\begin{scope}[xshift=2cm]
				\node[main node] (1) [fill=white] {};
				\node[main node] (2) [fill=white,dashed,below = 0.15cm of 1] {};
				
				\node (t1) [right = 0.2cm of 1] {homogene Zelle};
				\node (t2) [right = 0.2cm of 2] {heterogene Zelle};
				
				\node[bezier] (31) [below left = 0.55cm and 0.15cm of 2] {};
				\node[bezier] (32) [right = 0.85cm of 31] {};
				\path[draw,thick] (31) edge node {} (32);
				
				\node[bezier] (41) [below = 0.75cm of 31] {};
				\node[bezier] (42) [right = 0.85cm of 41] {};
				\path[draw,thick,dashed] (41) edge node {} (42);
				
				\node (t3) [right = 0.12cm of 32] {isotrope Kante};
				\node (t4) [right = 0.12cm of 42] {anisotrope Kante};
				
				\node[main node] (5) [fill = white, below = 1.6cm of 2] {$x$};
				\node (t5) [right = 0.2cm of 5] {Kardinalität $x$};
			\end{scope}
		\end{tikzpicture}
	\end{frame}
	\begin{frame}{Baumartige Struktur von CR-Graphen}
		\begin{Lemma}
			Angenommen ein CR-Graph $G$ erfüllt die Bedingungen \textbf{A-F}.
			Für jede anisotrope Komponente $A$ von $C(G)$ gelten folgende Eigenschaften:
			
			\begin{enumerate}[label=(\Alph*)]
				\setcounter{enumi}{6}
				\item $A$ ist ein Baum, der folgende Monotonieeigenschaft erfüllt: Sei $R$ eine Zelle aus $A$ mit minimaler Kardinalität, so ist $A_R$ der gerichtete Baum mit Wurzel $R$; Für jede gerichtete Kante $(X,Y)$ aus $A_R$ gilt dann $|X|\leq |Y|$.
				\item $A$ enthält maximal eine heterogene Zelle; Wenn eine solche Zelle existiert, hat diese minimale Kardinalität in $A$.
			\end{enumerate}
			\label{lemma:global2}
		\end{Lemma}
	\end{frame}
	\begin{frame}{Baumartige Struktur von CR-Graphen}
		\centering
		\textit{$A$ enthält maximal eine heterogene Zelle; Wenn eine solche Zelle existiert, hat diese minimale Kardinalität in $A$.}
		
		\begin{tikzpicture}
			\only<1>{\begin{scope}
				\node[main node] (1) [fill=blue!50] {$2$};
				\node[main node] (2) [fill=green!50,below left = 0.5cm and 0.85cm of 1] {$2$};
				\node[main node] (3) [dashed,fill=red!50,below = 1cm of 1] {$4$};
				
				\path[draw,thick,dashed] (1) edge node {} (2);
				\path[draw,thick,dashed] (2) edge node {} (3);
				\path[draw,thick] (1) edge node {} (3);
			\end{scope}
			\begin{scope}[xshift=2cm]
				\node[main node] (1) [fill=white] {};
				\node[main node] (2) [fill=white,dashed,below = 0.15cm of 1] {};
				
				\node (t1) [right = 0.2cm of 1] {homogene Zelle};
				\node (t2) [right = 0.2cm of 2] {heterogene Zelle};
				
				\node[bezier] (31) [below left = 0.55cm and 0.15cm of 2] {};
				\node[bezier] (32) [right = 0.85cm of 31] {};
				\path[draw,thick] (31) edge node {} (32);
				
				\node[bezier] (41) [below = 0.75cm of 31] {};
				\node[bezier] (42) [right = 0.85cm of 41] {};
				\path[draw,thick,dashed] (41) edge node {} (42);
				
				\node (t3) [right = 0.12cm of 32] {isotrope Kante};
				\node (t4) [right = 0.12cm of 42] {anisotrope Kante};
				
				\node[main node] (5) [fill = white, below = 1.6cm of 2] {$x$};
				\node (t5) [right = 0.2cm of 5] {Kardinalität $x$};
			\end{scope}}
			
			\only<2>{\begin{scope}
				\node[main node] (1) [fill=black!20] {$2$};
				\node[main node] (2) [fill=green!50,below left = 0.5cm and 0.85cm of 1] {$2$};
				\node[main node] (3) [dashed,fill=red!50,below = 1cm of 1] {$4$};
				
				\path[draw,thick,dashed,draw=black!30] (1) edge node {} (2);
				\path[draw,thick,dashed] (2) edge node {} (3);
				\path[draw,thick,draw=black!30] (1) edge node {} (3);
			\end{scope}
			\begin{scope}[xshift=2cm]
				\node[main node] (1) [fill=white] {};
				\node[main node] (2) [fill=white,dashed,below = 0.15cm of 1] {};
				
				\node (t1) [right = 0.2cm of 1] {homogene Zelle};
				\node (t2) [right = 0.2cm of 2] {heterogene Zelle};
				
				\node[bezier] (31) [below left = 0.55cm and 0.15cm of 2] {};
				\node[bezier] (32) [right = 0.85cm of 31] {};
				\path[draw,thick] (31) edge node {} (32);
				
				\node[bezier] (41) [below = 0.75cm of 31] {};
				\node[bezier] (42) [right = 0.85cm of 41] {};
				\path[draw,thick,dashed] (41) edge node {} (42);
				
				\node (t3) [right = 0.12cm of 32] {isotrope Kante};
				\node (t4) [right = 0.12cm of 42] {anisotrope Kante};
				
				\node[main node] (5) [fill = white, below = 1.6cm of 2] {$x$};
				\node (t5) [right = 0.2cm of 5] {Kardinalität $x$};
			\end{scope}
			\begin{scope}[yshift=-3cm,xshift=-1cm]
				\node {\Huge\Lightning};
			\end{scope}}
		\end{tikzpicture}
	\end{frame}
	
	\section{Was bringt mir jetzt das ganze?}
	\begin{frame}{Hinreichende Bedingung für CR-Graphen}
		\begin{Theorem}
			Für einen Graphen $G$ sind folgende Aussagen äquivalent:
			
			\begin{enumerate}[label=(\alph*)]
				\item $G$ ist ein CR-Graph
				\item $G$ erfüllt Bedingungen \emph{A-F}
				\item $G$ erfüllt Bedingungen \emph{A}, \emph{B}, \emph{G} und \emph{H}
			\end{enumerate}
		\end{Theorem}
		\pause
		
		\alert{\Large{\textit{Wir können also jetzt CR-Graphen erkennen!}}}
	\end{frame}
	\begin{frame}{Laufzeit}
		Für das Erkennen von CR-Graphen dominiert die Laufzeit zum Berechnen der stabilen Partition.
		
		$\Rightarrow$ $O((n+m)\log n)$
	\end{frame}
	\begin{frame}{Haupterkenntnis}
		\begin{itemize}
			\item Für die Klasse der CR-Graphen lässt sich das Isomorphieproblem mit dem Color-Refinement \alert{\textbf{in polynomieller Zeit}} lösen.
			\item Es kann \alert{\textbf{in polynomieller Zeit}} entschieden werden, ob ein Graph ein CR-Graph ist.
		\end{itemize}
	\end{frame}
\end{document}
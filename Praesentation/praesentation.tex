\documentclass{beamer}
\usetheme{metropolis}

\usepackage[utf8]{inputenc}
\usepackage[ngerman]{babel}                                 % deutsche Sprache
\usepackage[T1]{fontenc}                                    % Unterstützung für Umlaute mit Fonts
\usepackage[autostyle=true,german=quotes]{csquotes}			% Wird für den Befehl \enquote benötigt
\usepackage{tikz}											% Ermöglicht das Zeichnen von Grafiken
\usepackage{amssymb}										% Einfügen von zusätzlichen Symbolen wie \square
\usepackage{fontspec}

\setsansfont{Calibri}
\setmonofont{Courier New}

\metroset{titleformat=smallcaps,		% Überschriften werden in smallcaps geschrieben
	sectionpage=simple,					% Entfernen der Progressbar auf den Sectionpages
	block=fill,							% Hinterlegen von Umgebungen wie Theorem oder Example
}							

% Treffen von Einstellungen für tikz
\usetikzlibrary{positioning}
\usetikzlibrary{fit}
\tikzset{main node/.style={circle,fill=blue!30,draw,minimum size=0.7cm,inner sep=0pt},}
\tikzset{bezier/.style={inner sep=0pt,minimum size=0cm},}

\title{On the Power of Color Refinement}
\subtitle{V. Arvind, Johannes Köbler, Gaurav Rattan und\\ Oleg Verbitsky}
\date{05.02.2018}
\author{Florian Lüdiger}
\institute{Seminar Algorithm Engineering - Lehrstuhl 11 - TU Dortmund}
\begin{document}
	\maketitle
	
	\section{Wiederholung: Graph-Isomorphie und Color-Refinement}
	\begin{frame}
		\begin{itemize}
			\item Beispiel für GI
			\item Beispiel für CR
			\item Kernergebnis von CR
			\item Problem: nicht-isomorphe Graphen können nicht immer unterschieden werden (Beispiel)
		\end{itemize}
	\end{frame}

	\section{Was gibt es neues?}
	\begin{frame}
		\begin{itemize}
			\item Definition CR-Graph
		\end{itemize}
	\end{frame}

	\section{Lokale Struktur}

	\section{Globale Struktur}
	
	\section{Ergebnis}
	\begin{frame}
		\begin{itemize}
			\item Anwendung der vorgestellten Bedingungen
			\item Anwendungsbeispiel
		\end{itemize}
	\end{frame}

	\section{Backup-Folien}
	\begin{frame}
		\begin{itemize}
			\item Beweis lokale Struktur
			\item Ein Beweis für globale Struktur beispielhaft
		\end{itemize}
	\end{frame}
\end{document}
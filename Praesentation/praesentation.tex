\documentclass{beamer}
\usetheme{metropolis}

\usepackage[utf8]{inputenc}
\usepackage[ngerman]{babel}                                 % deutsche Sprache
\usepackage[T1]{fontenc}                                    % Unterstützung für Umlaute mit Fonts
\usepackage[autostyle=true,german=quotes]{csquotes}			% Wird für den Befehl \enquote benötigt
\usepackage{tikz}											% Ermöglicht das Zeichnen von Grafiken
\usepackage{amssymb}										% Einfügen von zusätzlichen Symbolen wie \square
\usepackage{fontspec}

% Einstellen von Formatierungen
\setsansfont{Calibri}
\setmonofont{Courier New}

\metroset{titleformat=smallcaps,		% Überschriften werden in smallcaps geschrieben
	sectionpage=simple,					% Entfernen der Progressbar auf den Sectionpages
	block=fill,							% Hinterlegen von Umgebungen wie Theorem oder Example
}							

% Treffen von Einstellungen für tikz
\usetikzlibrary{positioning}
\usetikzlibrary{fit}
\usetikzlibrary{shapes.geometric}
\tikzset{main node/.style={circle,fill=blue!50,draw,minimum size=0.7cm,inner sep=0pt},}
\tikzset{bezier/.style={inner sep=0pt,minimum size=0cm},}

% Erstellen eines Commands zum verringern der Seitenränder für eine Folie
\newcommand\Wider[2][3em]{%
	\makebox[\linewidth][c]{%
		\begin{minipage}{\dimexpr\textwidth+#1\relax}
			\raggedright#2
		\end{minipage}%
	}%
}

% Erstellen eines Commands zum Zeichnen von Zahlen mit einem Kreis darum
\newcommand\circlearound[1]{%
	\tikz[baseline]\node[draw,shape=circle,anchor=base] {#1} ;}

\newtheorem{Korollar}{Korollar}

\title{On the Power of Color Refinement}
\subtitle{V. Arvind, Johannes Köbler, Gaurav Rattan und\\ Oleg Verbitsky}
\date{05.02.2018}
\author{Florian Lüdiger}
\institute{Seminar Algorithm Engineering - Lehrstuhl 11 - TU Dortmund}
\begin{document}
	\maketitle
	
	\section{Wiederholung: Graph-Isomorphie und Color-Refinement}
	\begin{frame}
		\begin{itemize}
			\item Beispiel für GI
			\item Beispiel für CR
			\item Kernergebnis von CR
			\item Problem: nicht-isomorphe Graphen können nicht immer unterschieden werden (Beispiel)
		\end{itemize}
	\end{frame}
	\begin{frame}{Graph-Isomorphie}
		\begin{Definition}
			 Zwei Graphen $G$ und $H$ sind isomorph, wenn es eine bijektive Abbildung $\phi$ gibt, sodass gilt:\\ $(u,v)\in E_G\Leftrightarrow (\phi (u),\phi (v))\in E_H$ für alle $u,v\in V_G$.
		\end{Definition}
		% Vortragsnotizen:
		% Adjazenz
		% Isomorphismus
		% kein Polynomialzeitalgorithmus bekannt
	\end{frame}
	\begin{frame}{Beispiel}
		\centering
		\begin{tikzpicture}
			\begin{scope}
				\node[draw=none,minimum size=4cm,regular polygon,regular polygon sides=5] (a) {};
				\foreach \x in {1,2,...,5}
				\node[main node] at (a.corner \x) (\x) {};
				
				\path[draw,thick]
				(1) edge node {} (2)
				(2) edge node {} (3)
				(3) edge node {} (4)
				(4) edge node {} (5)
				(5) edge node {} (1);
			\end{scope}
			
			\begin{scope}[xshift=6cm]
				\node[draw=none,minimum size=4cm,regular polygon,regular polygon sides=5] (a) {};
				\foreach \x in {1,2,...,5}
					\node[main node] at (a.corner \x) (\x) {};
				
				\path[draw,thick]
				(1) edge node {} (4)
				(4) edge node {} (2)
				(2) edge node {} (5)
				(5) edge node {} (3)
				(3) edge node {} (1);
			\end{scope}
		\end{tikzpicture}
	\end{frame}
	\begin{frame}{Beispiel}
		\centering
		\begin{tikzpicture}
			\begin{scope}
				\node[draw=none,minimum size=4cm,regular polygon,regular polygon sides=5] (a) {};
				\foreach \x in {1,2,...,5}
					\node[main node] at (a.corner \x) (\x) {\x};
				
				\path[draw,thick]
				(1) edge node {} (2)
				(2) edge node {} (3)
				(3) edge node {} (4)
				(4) edge node {} (5)
				(5) edge node {} (1);
			\end{scope}
			
			\begin{scope}[xshift=6cm]
				\node[draw=none,minimum size=4cm,regular polygon,regular polygon sides=5] (a) {};
				\foreach \x in {1,2,...,5}
					\node[main node] at (a.corner \x) (\x) {\x};
				\node[main node] at (a.corner 1) (1) {1};
				\node[main node] at (a.corner 2) (2) {4};
				\node[main node] at (a.corner 3) (3) {2};
				\node[main node] at (a.corner 4) (4) {5};
				\node[main node] at (a.corner 5) (5) {3};
				
				\path[draw,thick]
				(1) edge node {} (4)
				(4) edge node {} (2)
				(2) edge node {} (5)
				(5) edge node {} (3)
				(3) edge node {} (1);
			\end{scope}
		\end{tikzpicture}
	\end{frame}
	\begin{frame}{Color-Refinement}
		\begin{Definition}
			Mit der Color-Refinement-Heuristik kann in polynomieller Zeit festgestellt werden, dass zwei Graphen nicht isomorph sind.
		\end{Definition}
		Anders gesagt gilt für beliebige Graphen $G$,$H$:\\
		\circlearound{1} CR unterscheidet $G$ und $H$ $\Rightarrow$ $G\not\simeq H$
	\end{frame}
	\begin{frame}{Beispiel}
		\Wider[4em]{
			\begin{tikzpicture}
		% BILD 1:
				\begin{scope}
					\node[main node] (1) {};
					\node[main node] (2) [right = 1cm of 1] {};
					\node[main node] (3) [below = 1cm of 1] {};
					\node[main node] (4) [right = 1cm of 3] {};
					\node[main node] (5) [below right = 0.35cm and 0.35cm of 1] {};
					\node (text) [left = 0cm of 1] {1)};
					
					\path[draw,thick]
					(1) edge node {} (2)
					(1) edge node {} (3)
					(3) edge node {} (4)
					(2) edge node {} (4)
					(1) edge node {} (5);
				\end{scope}
				
				\begin{scope}[xshift=3cm]
					\node[main node] (1) {};
					\node[main node] (2) [right = 1cm of 1] {};
					\node[main node] (3) [below = 1cm of 1] {};
					\node[main node] (4) [right = 1cm of 3] {};
					\node[main node] (5) [below right = 0.35cm and 0.35cm of 1] {};
					
					\path[draw,thick]
					(1) edge node {} (3)
					(3) edge node {} (5)
					(2) edge node {} (5)
					(2) edge node {} (4)
					(4) edge node {} (5);
				\end{scope}
				\pause
		% BILD 2:
				\begin{scope}[yshift=-3cm]
					\node[main node] (1) [fill=red!50] {};
					\node[main node] (2) [right = 1cm of 1] {};
					\node[main node] (3) [below = 1cm of 1] {};
					\node[main node] (4) [right = 1cm of 3] {};
					\node[main node] (5) [fill=green!50,below right = 0.35cm and 0.35cm of 1] {};
					\node (text) [left = 0cm of 1] {2)};
					
					\path[draw,thick]
					(1) edge node {} (2)
					(1) edge node {} (3)
					(3) edge node {} (4)
					(2) edge node {} (4)
					(1) edge node {} (5);
				\end{scope}
				
				\begin{scope}[yshift=-3cm,xshift=3cm]
					\node[main node] (1) [fill=green!50]{};
					\node[main node] (2) [right = 1cm of 1] {};
					\node[main node] (3) [below = 1cm of 1] {};
					\node[main node] (4) [right = 1cm of 3] {};
					\node[main node] (5) [fill=red!50,below right = 0.35cm and 0.35cm of 1] {};
					
					\path[draw,thick]
					(1) edge node {} (3)
					(3) edge node {} (5)
					(2) edge node {} (5)
					(2) edge node {} (4)
					(4) edge node {} (5);
				\end{scope}
				\pause
		% BILD 3:
				\begin{scope}[xshift=6.2cm]
					\node[main node] (1) [fill=red!50] {};
					\node[main node] (2) [right = 1cm of 1] {};
					\node[main node] (3) [below = 1cm of 1] {};
					\node[main node] (4) [fill=yellow!50,right = 1cm of 3] {};
					\node[main node] (5) [fill=green!50,below right = 0.35cm and 0.35cm of 1] {};
					\node (text) [left = 0cm of 1] {3)};
					
					\path[draw,thick]
					(1) edge node {} (2)
					(1) edge node {} (3)
					(3) edge node {} (4)
					(2) edge node {} (4)
					(1) edge node {} (5);
				\end{scope}
				
				\begin{scope}[xshift=9.2cm]
					\node[main node] (1) [fill=orange!50]{};
					\node[main node] (2) [right = 1cm of 1] {};
					\node[main node] (3) [fill=pink!50,below = 1cm of 1] {};
					\node[main node] (4) [right = 1cm of 3] {};
					\node[main node] (5) [fill=black!50,below right = 0.35cm and 0.35cm of 1] {};
					
					\path[draw,thick]
					(1) edge node {} (3)
					(3) edge node {} (5)
					(2) edge node {} (5)
					(2) edge node {} (4)
					(4) edge node {} (5);
				\end{scope}
				\pause
		% BILD 4:
				\begin{scope}[yshift=-3cm,xshift=6.2cm]
					\node[main node] (1) [fill=red!50] {};
					\node[main node] (2) [right = 1cm of 1] {};
					\node[main node] (3) [below = 1cm of 1] {};
					\node[main node] (4) [fill=yellow!50,right = 1cm of 3] {};
					\node[main node] (5) [fill=green!50,below right = 0.35cm and 0.35cm of 1] {};
					\node (text) [left = 0cm of 1] {4)};
					
					\path[draw,thick]
					(1) edge node {} (2)
					(1) edge node {} (3)
					(3) edge node {} (4)
					(2) edge node {} (4)
					(1) edge node {} (5);
				\end{scope}
				
				\begin{scope}[yshift=-3cm,xshift=9.2cm]
					\node[main node] (1) [fill=orange!50]{};
					\node[main node] (2) [fill=white!50,right = 1cm of 1] {};
					\node[main node] (3) [fill=pink!50,below = 1cm of 1] {};
					\node[main node] (4) [fill=white!50,right = 1cm of 3] {};
					\node[main node] (5) [fill=black!50,below right = 0.35cm and 0.35cm of 1] {};
					
					\path[draw,thick]
					(1) edge node {} (3)
					(3) edge node {} (5)
					(2) edge node {} (5)
					(2) edge node {} (4)
					(4) edge node {} (5);
				\end{scope}
			\end{tikzpicture}
		}
	\end{frame}
	\begin{frame}{Limitierung der Heuristik}
		Es gibt nicht-isomorphe Graphenpaare, welche das Color-Refinement nicht unterscheiden kann.
		
	
		\begin{tikzpicture}
			\begin{scope}
				\node[main node] (1) {};
				\node[main node] (2) [below left = 0.8cm and 1.4cm of 1]  {};
				\node[main node] (6) [below right = 0.8cm and 1.4cm of 1] {};
				\node[main node] (3) [below = 1.4cm of 2] {};
				\node[main node] (5) [below = 1.4cm of 6] {};
				\node[main node] (4) [below right = 0.8cm and 1.4cm of 3] {};
				
				\path[draw,thick]
				(1) edge node {} (2)
				(2) edge node {} (3)
				(3) edge node {} (4)
				(4) edge node {} (5)
				(5) edge node {} (6)
				(6) edge node {} (1);
			\end{scope}
			
			\node [below right = 0.4cm and 0.8cm of 6] {\huge $\not\simeq$};
			
			\begin{scope}[xshift=6.5cm]
				\node[main node] (1) {};
				\node[main node] (2) [below left = 0.8cm and 1.4cm of 1]  {};
				\node[main node] (6) [below right = 0.8cm and 1.4cm of 1] {};
				\node[main node] (3) [below = 1.4cm of 2] {};
				\node[main node] (5) [below = 1.4cm of 6] {};
				\node[main node] (4) [below right = 0.8cm and 1.4cm of 3] {};
				
				\path[draw,thick]
				(1) edge node {} (2)
				(2) edge node {} (6)
				(3) edge node {} (4)
				(4) edge node {} (5)
				(5) edge node {} (3)
				(6) edge node {} (1);
			\end{scope}
		\end{tikzpicture}
	\end{frame}

	\section{Was gibt es neues?}
	\begin{frame}
		\begin{itemize}
			\item Definition CR-Graph
		\end{itemize}
	\end{frame}
	\begin{frame}{Die Klasse der CR-Graphen}
		\begin{Definition}
			Graph $G$ ist \textbf{CR-Graph}, wenn das Color-Refinement diesen von jedem nicht zu $G$ isomorphen Graphen $H$ unterscheiden kann.
		\end{Definition}
		Für beliebige CR-Graphen $G$,$H$ gilt also:\\
		\circlearound{2} $G\not\simeq H$ $\Rightarrow$ CR unterscheidet $G$ und $H$
	\end{frame}
	\begin{frame}{Ergebnis und Beobachtung}
		\circlearound{1} CR unterscheidet $G$ und $H$ $\Rightarrow$ $G\not\simeq H$\\
		\circlearound{2} $G\not\simeq H$ $\Rightarrow$ CR unterscheidet $G$ und $H$
		\begin{Korollar}
			Für zwei CR-Graphen $G$ und $H$ gilt:\\
			CR erkennt $G$ und $H$ als isomorph $\Leftrightarrow$ $G\simeq H$
		\end{Korollar}
		\pause
		\vspace{10mm}
		\Large{\textit{Wie identifiziere ich also die Klasse der CR-Graphen?}}
	\end{frame}

	\section{Lokale Struktur}

	\section{Globale Struktur}
	
	\section{Ergebnis}
	\begin{frame}
		\begin{itemize}
			\item Anwendung der vorgestellten Bedingungen
			\item Anwendungsbeispiel
		\end{itemize}
	\end{frame}

	\section{Backup-Folien}
	\begin{frame}
		\begin{itemize}
			\item Beweis lokale Struktur
			\item Ein Beweis für globale Struktur beispielhaft
		\end{itemize}
	\end{frame}
\end{document}
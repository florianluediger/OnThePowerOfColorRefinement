\section{Ergebnis und Laufzeit}
\label{sec/ergebnis}

%\begin{itemize}
%	\item Erklärung von Theorem 9
%	\item Vorstellung des Ergebnisses
%	\item Beweis der Laufzeit von $O((n+m)\log n)$
%\end{itemize}

Die vorgestellten lokalen und globalen Eigenschaften von CR-Graphen reichen wie im Folgenden gezeigt aus, um hinreichende Bedingungen für CR-Graphen zu formulieren und darauf basierend ein effizientes Verfahren für das Erkennen solcher Graphen zu entwickeln.

\subsection{Hinreichende Bedingungen für das Erkennen von CR-Graphen}
\begin{Theorem}
	Für einen Graphen $G$ sind folgende Aussagen äquivalent:
	
	\begin{enumerate}[label=(\alph*)]
		\item $G$ ist ein CR-Graph
		\item $G$ erfüllt Bedingungen \emph{A-F}
		\item $G$ erfüllt Bedingungen \emph{A}, \emph{B}, \emph{G} und \emph{H}
	\end{enumerate}
\end{Theorem}

\emph{Beweis:} Die Äquivalenz der Aussagen wird gezeigt, indem gezeigt wird, dass gilt: $(a)\rightarrow (b)\rightarrow (c)\rightarrow (a)$.
Die bisher erlangten Erkenntnisse ermöglichen es bereits einen großen Teil dieser Aussage zu bestätigen. Somit wurde in den Lemmata \ref{lemma:lokal} und \ref{lemma:global1} gezeigt, dass $(a)\rightarrow (b)$ gilt. Ebenfalls wurde in Lemma \ref{lemma:global2} gezeigt, dass $(b)\rightarrow (c)$ gilt. Es bleibt also nur noch zu zeigen, dass auch $(c)\rightarrow (a)$ gültig ist.

[Beweis folgt]

\subsection{Laufzeit}
Zur Berechnung der Laufzeit wird im Folgenden davon ausgegangen, dass der Graph $G$ in Adjazenzlistendarstellung vorliegt.
Nach \cite{CARDON198285} lässt sich die stabile Partition eines Graphen $G$ in Zeit $\mathcal{O}((n+m)\log n)$ berechnen.

\begin{Theorem}
	Die Klasse der CR-Graphen ist in Zeit $\mathcal{O}((n+m)\log n)$ entscheidbar. Dabei bezeichnet $n$ die Anzahl der Knoten und $m$ die Anzahl der Kanten des Eingabegraphen.
\end{Theorem}

\emph{Beweis:} Zunächst wird die stabile Partition $\mathcal{P}_G=\{X_1,X_2,...,X_k\}$ berechnet, was wie eingangs erwähnt die Laufzeit $\mathcal{O}((n+m)\log n)$ benötigt.
Außerdem wird $C^*(G)$ definiert als der Zellgraph, bei dem sämtliche leeren Kanten, also solche bei denen keine Verbindungen zwischen den Elementen der beiden Endpunkte besteht, entfernt wurden.

Für die Elemente $X_i\in C^*(G)$ werden die Adjazenzlisten gebildet, indem die Adjazenzliste eines beliebigen Knoten $u\in X_i$ durchlaufen wird und sämtliche Zellen aufgelistet werden, welche einen zu $u$ adjazenten Knoten enthalten.
Die dadurch gewonnenen Informationen sind identisch für alle Knoten aus $X_i$, da diese alle gleichartige Nachbarschaften besitzen, weshalb es ausreicht die Operation für einen beliebigen Knoten durchzuführen.
Durch die Informationen aus der Adjazenzliste lässt sich der Grad der Knoten innerhalb der Zellen bestimmen und somit leicht Bedingung \emph{A} durch die in Lemma \ref{lemma:lokal_regulaer} vorgestellten Bedingungen für jeden Subgraphen $G[X_i]$ überprüfen.

Für jede Kante ${X_i,X_j}$ aus $C^*(G)$ wird der Wert $d_{ij}$ berechnet, welcher die Anzahl der Nachbarn in $X_j$ beschreibt, zu denen jeder Knoten aus $X_i$ adjazent ist.
Dieser Wert wird ebenfalls für den Fall $i=j$ betrachtet, wobei die Nachbarn innerhalb der Zelle gezählt werden.
Dadurch, dass die Werte $|X_i|$, $|X_j|$ und $d_{ij}$ nun bekannt sind, lässt sich Bedingung \emph{B} durch die in Lemma \ref{lemma:lokal_bipartit} vorgestellten Bedingungen überprüfen.

Da alle Zellen des Zellgraphen $C^*(G)$ im schlimmsten Falle unterschiedliche Farben haben, gibt es maximal $n$ Zellen.
Mit dem beschriebenen Verfahren können die Bedingungen \emph{A} und \emph{B} somit in der Zeit $\mathcal{O}(n)$ überprüft werden.

Zum Überprüfen von Bedingung \emph{H} wird eine Breitensuche auf dem Zellgraphen $C^*(G)$ durchgeführt, welche alle anisotropen Komponenten findet und gleichzeitig überprüft, ob diese eine Baumstruktur aufweisen und nur eine anisotrope Komponente enthält.

Wird diese Breitensuche nun von der Zelle minimaler Kardinalität für jede Komponente wiederholt, so lässt sich die in Bedingung \emph{G} beschriebene Monotonieeigenschaft für jede Kante überprüfen.

Der Breitensuchealgorithmus benötigt eine Laufzeit von $\mathcal{O}(n+m)$, da der Zellgraph maximal $n$ Knoten und maximal $m$ Kanten besitzt.

Es ist somit zu erkennen, dass das Errechnen der stabilen Partition, auf der die ganzen Operationen ausgeführt werden mit $\mathcal{O}((n+m)\log n)$ die dominierende Laufzeit des Algorithmus ist.
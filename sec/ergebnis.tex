\section{Ergebnis und Laufzeit}
\label{sec/ergebnis}

%\begin{itemize}
%	\item Erklärung von Theorem 9
%	\item Vorstellung des Ergebnisses
%	\item Beweis der Laufzeit von $O((n+m)\log n)$
%\end{itemize}

Die vorgestellten lokalen und globalen Eigenschaften von CR-Graphen reichen wie im Folgenden gezeigt aus, um hinreichende Bedingungen für CR-Graphen zu formulieren und darauf basierend ein effizientes Verfahren für das Erkennen solcher Graphen zu entwickeln.

\subsection{Hinreichende Bedingungen für das Erkennen von CR-Graphen}
\begin{Theorem}
	Für einen Graphen $G$ sind folgende Aussagen äquivalent:
	
	\begin{enumerate}[label=(\alph*)]
		\item $G$ ist ein CR-Graph
		\item $G$ erfüllt Bedingungen \emph{A-F}
		\item $G$ erfüllt Bedingungen \emph{A}, \emph{B}, \emph{G} und \emph{H}
	\end{enumerate}
\end{Theorem}

\emph{Beweis:} Die Äquivalenz der Aussagen wird gezeigt, indem gezeigt wird, dass gilt: $(a)\rightarrow (b)\rightarrow (c)\rightarrow (a)$.
Die bisher erlangten Erkenntnisse ermöglichen es bereits einen großen Teil dieser Aussage zu bestätigen. Somit wurde in den Lemmata \ref{lemma:lokal} und \ref{lemma:global1} gezeigt, dass $(a)\rightarrow (b)$ gilt. Ebenfalls wurde in Lemma \ref{lemma:global2} gezeigt, dass $(b)\rightarrow (c)$ gilt. Es bleibt also nur noch zu zeigen, dass auch $(c)\rightarrow (a)$ gültig ist.

[Beweis folgt]

\subsection{Laufzeit}
Zur Berechnung der Laufzeit wird im Folgenden davon ausgegangen, dass der Graph $G$ in Adjazenzlistendarstellung vorliegt.
Nach \cite{CARDON198285} lässt sich die stabile Partition eines Graphen $G$ in Zeit $\mathcal{O}((n+m)\log n)$ berechnen.

\begin{Theorem}
	Die Klasse der CR-Graphen ist in Zeit $\mathcal{O}((n+m)\log n)$ entscheidbar. Dabei bezeichnet $n$ die Anzahl der Knoten und $m$ die Anzahl der Kanten des Eingabegraphen.
\end{Theorem}

\emph{Beweis:} [folgt]
\section{Globale Struktur von CR-Graphen}
\label{sec/struktur_global}

%\begin{itemize}
%	\item Erklärung der globalen Struktur zugänglicher Graphen (C,D,E,F,G,H)
%	\item Ansonten wie Kapitel \ref{sec/struktur_lokal}
%\end{itemize}

Zusätzlich zur lokalen Struktur ist außerdem die globale Struktur von CR-Graphen interessant.
Der Begriff globale Struktur gibt hier an, dass der Zellgraph von $G$ bestimmte Eigenschaften erfüllen muss.

\begin{Definition}
	Der Zellgraph $C(G)$ eines Graphen $G$ wird aus dessen stabilen Partition $\mathcal{P}_G$ gebildet.
	Es handelt sich dabei um einen vollständigen Graphen, bei dem die Knoten die Zellen von $\mathcal{P}_G$ darstellen.
\end{Definition}

\subsection{Mögliche Eigenschaften von Zellgraphen}
Für das Verständnis der folgenden Eigenschaften von CR-Graphen ist es erforderlich einige Begriffe einzuführen und Eigenschaften von Zellgraphen zu erklären und zu benennen, wozu einige Definitionen folgen.
Die Knoten eines Zellgraphen, auch als Zellen bezeichnet, können dabei folgende Eigenschaften aufweisen.

\begin{Definition}
	Eine Zelle $X\in C(G)$ wird \emph{homogen} genannt, wenn der Graph $G[X]$ vollständig oder leer ist. Anderenfalls wird diese \emph{heterogen} genannt.
\end{Definition}

\begin{Definition}
	Für eine heterogene Zelle $X\in C(G)$ finden sich je nach Beschaffenheit von $G[X]$ die Bezeichnungen \emph{matching}, \emph{co-matching} oder \emph{pentagonal}.
	Eine homogene Zelle wird dagegen entweder \emph{leer} oder \emph{vollständig} genannt.
\end{Definition}

Für die Kanten eines Zellgraphen finden sich ebenfalls unterschiedliche Bezeichnungen, welche deren Beschaffenheit beschreiben.

\begin{Definition}
	Eine Kante ${X,Y}$ mit $X,Y\in C(G)$ wird \emph{isotrop} genannt, wenn der bipartite Graph $G[X,Y]$ vollständig oder leer ist. Anderenfalls wird diese \emph{anisotrop} genannt.
\end{Definition}

\begin{Definition}
	Eine anisotrope Kante ${X,Y}$ wird \emph{Konstellation} genannt, wenn $G[X,Y]$ eine disjunkte Vereinigung von Sternen ist.
	Anderenfalls wird diese \emph{Co-Konstellation} genannt.
	Bei Co-Konstellationen bildet das \glslink{bipartites_komplement}{bipartite Komplement} von $G[X,Y]$ eine disjunkte Vereinigung von Sternen.
	Eine isotrope Kante dagegen wird entweder \emph{leer} oder \emph{vollständig} genannt.
\end{Definition}

\begin{Definition}
	 Ein Pfad $X_1X_2...X_l$ in $C(G)$, bei dem jede Kante ${X_i,X_{i+1}}$ anisotrop ist, wird \emph{anisotroper Pfad} genannt. Wenn dieser Pfad einen Kreis schließt, wird er als \emph{anisotroper Zyklus} bezeichnet. Gilt für einen anisotropen Pfad $|X_1|=|X_2|=...=|X_l|$ dann wird er \emph{gleichmäßig} genannt.
\end{Definition}

\subsection{Allgemeine globale Eigenschaften von CR-Graphen}
Mit dem so gewonnenen Hintergrundwissen kann nun das folgende Lemma formuliert werden, welches globale Eigenschaften von CR-Graphen definiert.

\begin{Lemma}
	Der Zellgraph $C(G)$ eines CR-Graphen $G$ erfüllt folgende Eigenschaften:

	\begin{enumerate}[label=(\Alph*)]
		\setcounter{enumi}{2}
		\item $C(G)$ enthält keinen gleichmäßigen, anisotropen Pfad, der zwei heterogene Zellen verbindet
		\item $C(G)$ enthält keinen gleichmäßigen, anisotropen Zyklus
		\item $C(G)$ enthält weder einen anisotropen Pfad $XY_1Y_2...Y_lZ$, sodass $|X|<|Y_1|=|Y_2|=...=|Y_l|>|Z|$, noch einen anisotropen Zyklus $XY_1Y_2...Y_l$, sodass $|X|<|Y_1|=|Y_2|=...=|Y_l|$ und die Zelle $Y_l$ heterogen ist
		\item $C(G)$ enthält keinen anisotropen Pfad $XY_1Y_2...Y_l$, sodass $|X|<|Y_1|=|Y_2|=...=|Y_l|$ und die Zelle $Y_l$ heterogen ist
	\end{enumerate}
	\label{lemma:global1}
\end{Lemma}

\subsection{Baumartige Struktur von CR-Graphen}
Bei genauerer Betrachtung fällt auf, dass der Zellgraph von CR-Graphen eine baumartige Struktur aufweist.
Um das folgende Lemma verstehen zu können, wird der Begriff \emph{anisotrope Komponente} benötigt.
\begin{Definition}
	In einem Zellgraphen $C(G)$ bezeichnet eine \emph{anisotrope Komponente} einen Subgraphen, dessen Kanten alle isotrop sind.
	Wenn eine Zelle keine inzidente, anisotrope Kante besitzt, dann ergibt sich daraus eine anisotrope Komponente mit nur einer Zelle.
\end{Definition}

\begin{Lemma}
	Angenommen ein CR-Graph $G$ erfüllt die Bedingungen \emph{A-F} aus den Lemmata \ref{lemma:lokal} und \ref{lemma:global1}.
	Für jede anisotrope Komponente $A$ von $C(G)$ gelten folgende Eigenschaften:
	
	\begin{enumerate}[label=(\Alph*)]
		\setcounter{enumi}{6}
		\item $A$ ist ein Baum, der folgende Monotonieeigenschaft erfüllt: Sei $R$ eine Zelle aus $A$ mit minimaler Kardinalität, so ist $A_R$ der gerichtete Baum mit Wurzel $R$; Für jede gerichtete Kante $(X,Y)$ aus $A_R$ gilt dann $|X|\leq |Y|$
		\item $A$ enthält maximal eine heterogene Zelle; Wenn eine solche Zelle existiert, hat diese minimale Kardinalität in $A$
	\end{enumerate}
	\label{lemma:global2}
\end{Lemma}

